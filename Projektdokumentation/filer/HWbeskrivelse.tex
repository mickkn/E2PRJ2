\begin{figure}[!htbp] \centering
\subsection{BDD Hardware}
{\includegraphics[width=0.9\textwidth]{billeder/diagrammer/BDD_Hardware}}
\caption{BDD Hardware}
\label{lab:bddhardware}
\raggedright
BDD diagrammet giver et overblik over hvad det samlede system består af. Vi ser en port beskrivelse som viser hvilke signaler hver blok består af.
\end{figure}
BDD diagrammet giver et overblik over hvad det samlede system består af. Vi ser en port beskrivelse som viser hvilke signaler hver blok består af.

\begin{figure}[!htbp] \centering
\subsection{BDD Hovedenhed}
{\includegraphics[width=0.9\textwidth]{billeder/diagrammer/BDD_Hovedenhed}}
\caption{BDD Hovedenhed}
\label{lab:bddhovedenhed}
\raggedright
BDD diagrammet giver et overblik over hvad CSS hovedenheden består af. Vi ser en portbeskrivelse for STK kittet og encoder samt et overblik over hvilke komponenter encoderen består af.
\end{figure}
BDD diagrammet giver et overblik over hvad CSS hovedenheden består af. Vi ser en portbeskrivelse for STK kittet og encoder samt et overblik over hvilke komponenter encoderen består af. 

\begin{figure}[!htbp] \centering
\subsection{BDD Modtager}
{\includegraphics[width=0.9\textwidth]{billeder/diagrammer/BDD_Modtager}}
\caption{BDD Modtager}
\label{lab:bddmodtager}
\raggedright
BDD diagrammet giver et overblik over hvad X10 modtageren består af. Vi ser en portbeskrivelse for STK kittet og decoderen samt et overblik over hvilke komponenter decoderen består af.
\end{figure}
BDD diagrammet giver et overblik over hvad X10 modtageren består af. Vi ser en portbeskrivelse for STK kittet og decoderen samt et overblik over hvilke komponenter decoderen består af.

\begin{figure}[!htbp] \centering
\subsection{IBD Hardware}
{\includegraphics[width=0.9\textwidth]{billeder/diagrammer/IBD_Hardware}}
\caption{IBD Hardware}
\label{lab:ibdhardware}
\raggedright
IBD diagrammet giver et internt overblik over hvordan hele vores system er forbundet. Vi ser hvilke type signaler der bliver sendt imellem vores forskellige blokke.
\end{figure}
IBD diagrammet giver et internt overblik over hvordan hele vores system er forbundet. Vi ser hvilke type signaler der bliver sendt imellem vores forskellige blokke.

\begin{figure}[!htbp] \centering
\subsection{IBD Hovedenhed og Modtager}
{\includegraphics[width=0.9\textwidth]{billeder/diagrammer/IBD_Hovedenhed_Modtager}}
\caption{IBD Hovedenhed og Modtager}
\label{lab:ibdhovedenhedmodtager}
\raggedright
IBD diagrammet giver et internt overblik over hvordan vores X10 hovedenhed og X10 modtager er forbundet. Vi ser hvilke type signaler der bliver sendt imellem vores forskellige blokke.
\end{figure}
IBD diagrammet giver et internt overblik over hvordan vores X10 hovedenhed og X10 modtager er forbundet. Vi ser hvilke type signaler der bliver sendt imellem vores forskellige blokke.

\subsection{Signaltabel}

\subsubsection{Hovedenhed}
\begin{tabular}{|p{3cm}|p{2,4cm}|p{2,4cm}|p{2,4cm}|p{2,4cm}|}
\hline 
\textbf{Funktion} &\textbf{Område} &\textbf{Signaltype} &\textbf{Terminal 1} &\textbf{Terminal 2} \\ 
\hline 
\multicolumn{5}{|l|}{\textbf{:Encoder}} \\ 
\hline 
X10 data &databit &data &HE1 &HSTK4\\ 
\hline 
Send kommando ud på 18 V nettet &18 VAC \newline 120KHz &?? &HE2 &18 VAC\\ 
\hline 
\multicolumn{5}{|l|}{\textbf{:STK500}} \\ 
\hline 
RS232 datainput &RS232 &?? &HSTK1 &Computer\\ 
\hline 
Password bool  &0-5 V TTL &bool &HSTK2 &DE2\\ 
\hline 
Høj ved lyddektektion &0-5 V TTL &bool &HSTK3 &Lyddektektor\\ 
\hline 
X10 data &databit &data &HSTK4 &HE1\\ 
\hline 
\end{tabular} 

<<<<<<< HEAD
\clearpage
\newpage

\begin{table}[htbp] %% Blok og Signal Tabel
\subsection{Grænseflade}
For at opnå forståelse for signaler mellem blokkene laves en grænseflade der beskriver de enkelte blokkes porte og hvilke signaler der løber mellem disse.

\subsubsection{Blok beskrivelse}
Til at beskrive blokkene nærmere er anvendt tabeller som ses herunder. Her er hvert signal i en respektiv blok kommenteret og blokkens funktion er kort beskrevet. 

\begin{tabular}{|p{3,3cm}|p{3,3cm}|p{3,3cm}|p{3,3cm}|}
\hline
\textbf{Bloknavn} & \textbf{Funktion} & \textbf{Signaler} & \textbf{Kommentar} \\ \hline

Encoder & modtage kommando og encode til 120 kHz bursts & 120 kHz & Data ud \\ \cline{3-4}	
& & X10 data & X10 data kommando ind \\ \hline

STK500 Hovedenhed & Genererer burst og detekterer på zero-crossing & RS232 & Laptop forbindelse \\ \cline{3-4}
& & X10 data & X10 data kommando linje \\ \cline{3-4}
& & Bool & lyd detektion \\ \cline{3-4}
& & Bool & Password accept \\ \hline

Decoder & Modtager 120 kHz og decoder til X10 data & 120 kHz & 120 kHz ind \\ \cline{3-4}
& & X10 data & Kommando linje \\ \cline{3-4}
& & 120 kHz & 120 kHz data ind \\ \hline

STK 500 Decoder & Modtager burst og detekterer på zero-crossing & X10 data & X10 data ind \\ \cline{3-4}
&& Bool & Power I/O ekstern enhed \\ \hline 
\end{tabular}

\subsubsection{Signal beskrivelse}
For at fuldende beskrivelsen af grænsefladen er der lavet en signaltabel som kan ses herunder. Hvert signal er beskrevet og tilknyttet en kort kommentar. Området et signal er defineret under er også beskrevet. Blok og terminal indgår også. 

\begin{tabular}{|p{2cm}|p{2cm}|p{2cm}|p{2cm}|p{2cm}|p{2,2cm}|}
\hline
\textbf{Signal-navn} & \textbf{Funktion} & \textbf{Område} & \textbf{Port 1} & \textbf{Port 2} & \textbf{Kommentar} \\ \hline

120 kHz & sende kommando på 18V nettet & & Encoder, HE2 & Decoder, DM1 & \\ \hline

X10 data & kommando & & STK500, HSTK4 & Encoder, HE1 & \\
&&& Decoder, MD2 & STK500, MSTK500 &\\ \hline
=======
\subsubsection{Modtager}
\begin{tabular}{|p{3cm}|p{2,4cm}|p{2,4cm}|p{2,4cm}|p{2,4cm}|}
\hline 
\textbf{Funktion} &\textbf{Område} &\textbf{Signaltype} &\textbf{Terminal 1} &\textbf{Terminal 2} \\ 
\hline 
\multicolumn{5}{|l|}{\textbf{:Decoder}} \\ 
\hline 
Modtager kommando fra 18 V nettet &18 VAC \newline 120KHz &?? &MD1 &18 VAC\\ 
\hline 
X10 data &databit &data &MD2 &MSTK1\\ 
\hline 
\multicolumn{5}{|l|}{\textbf{:STK500}} \\ 
\hline 
X10 data &databit &data &MSTK1 &MD2\\ 
\hline 
Kommando signal til udtag  &0-5 V TTL &bool &MSTK2 &Udtag\\ 
\hline 
>>>>>>> parent of fe63449... Git hub fucker. committer ALT i repos.

\end{tabular} 
\end{table}
