For CSS hovedenheden er udviklet en række diagrammer ud fra applikationsmodel metoden. De bygger viderer på domænemodellen for hele systemet.
Der er et sekvensdiagram på figur \ref{fig:CSS_hovedenhed_SD} for alle de aktuelle use-cases som beskriver systemets virkemåde.
Ud fra dette er der lavet et klassediagram på figur \ref{fig:CSS_hovedenhed_Class} som dækker de forskellige use-cases med controller klasser og kommunikationenen til PC via RS232 og X10 Udtag via X10.


\begin{figure}[!htb]
	\includegraphics[width=\textwidth]{billeder/uml/CSS_hovedenhed_SD}
     \caption{Use-case sekvensdiagrammer for CSS hovedenhed}
     \label{fig:CSS_hovedenhed_SD}
\end{figure}

\begin{figure}[!htb] \centering
     \includegraphics[width=0.8\textwidth]{billeder/uml/CSS_hovedenhed_Class}
     \caption{Klassediagram for CSS hovedenhed}
     \label{fig:CSS_hovedenhed_Class}
\end{figure}

Under designfasen og implementeringen laves der er række støtte metoder til klasserne. Det endelige klassediagram kan ses på  figur \ref{fig:CSS_hovedenhed_Class_Static}. De globale funktionaliteter som egentlig strider mod objekt orienteret programmering er resultatet af at arbejde med software til en microcontroller som har en række autonomo funktionaliteter som bliver eksekveret fra udefra kommende hardware signaler.

\begin{figure}[!htb] \centering
     \includegraphics[width=\textwidth]{billeder/uml/CSS_hovedenhed_Class_Static}
     \caption{Statisk klassediagram for CSS hovedenhed}
     \label{fig:CSS_hovedenhed_Class_Static}
\end{figure}