Med informationerne fra domænemodellen er der udviklet en række diagrammer ud fra applikationsmodellen.
Applikationsmodellen er en metode til at fin-tænke software designet inden man går i gang med implementeringen. Ud fra de aktuelle Usecases er funktionaliteten beskrevet med sekvensdiagrammer i figur \ref{fig:PC_SD1} og figur \ref{fig:PC_SD2}.

\begin{figure}[H] \centering
     {\includegraphics[width=0.9\textwidth]{billeder/uml/PC_SD1}}
     \caption{Use-case 1-3 sekvensdiagram for PC}
     \label{fig:PC_SD1}
\end{figure}
\clearpage

\begin{figure}[H] \centering
     {\includegraphics[width=0.9\textwidth]{billeder/uml/PC_SD2}}
     \caption{Use-case 5-8 sekvensdiagram for PC}
     \label{fig:PC_SD2}
\end{figure}

\clearpage

Alle kaldene til controller-klasserne kommer fra main.cpp programmet som tager imod brugerinputs i preLogin menuen og mainMenuen. Når mainMenuen er vist så står main.cpp og spørger på read() metoden i RS232IF klassen som tester om den har modtaget data fra STK500-kittet. Dette gør den for at se om login status har ændret sig eller om der skal sendes en sms til brugeren pga. babyalarmen.\\
Metode navnene i controller-klasserne kan ses på næste sidste i klassediagrammet som er lavet på baggrund af Sekvens-diagrammerne.

\begin{figure}[H]
     {\includegraphics[width=\textwidth]{billeder/uml/PC_Class}}
     \caption{Klassediagram for PC}
     \label{fig:PC_Class}
\end{figure}

\clearpage
\vspace*{30 px}
\begin{figure}[H]
     {\includegraphics[width=\textwidth]{billeder/uml/PC_Class_static}}
     \caption{Statisk klassediagram for PC}
     \label{fig:PC_Class_Static}
\end{figure}

Main opretter alle objekterne og pointerne til de klasser hvis constructor skal bruge dem. Derudover styre main hvilke controllers der bliver kaldt alt efter bruger input.
%
%\begin{figure}[!htb]
%     \xput[0.471]{\includegraphics[width=1.32\linewidth]{billeder/uml/PC_SD1}}
%     \caption{Use-case 1-3 sekvensdiagram for PC}
%     \label{fig:PC_SD1}
%\end{figure}
