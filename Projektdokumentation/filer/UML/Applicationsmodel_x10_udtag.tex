For X10-udtag er der udviklet en række diagrammer ud fra applikationsmodel metoden.
På figur \ref{fig:X10_udtag_domain} er domænemodellen som resten af diagrammerne er udviklet ud fra.
Der er et sekvensdiagram på figur \ref{fig:X10_udtag_SD} for alle de aktuelle use-cases som beskriver systemets virkemåde.
Ud fra dette er der lavet et klassediagram på figur \ref{fig:X10_udtag_Class} som dækker de forskellige use-cases med controller klasser og kommunikationen til CSS hovedenheden via X10.


\begin{figure}[!htb] \centering
\centering \includegraphics[width=0.6\textwidth]{billeder/uml/X10_modtager_Domain}
     \caption{Domænemodel for X10-udtag}
     \label{fig:X10_udtag_domain}
\end{figure}

\begin{figure}[!htb]
	\includegraphics[width=\textwidth]{billeder/uml/X10_modtager_SD}
     \caption{Use-case sekvensdiagrammer for X10-udtag}
     \label{fig:X10_udtag_SD}
\end{figure}

\begin{figure}[!htb]
     \includegraphics[width=\textwidth]{billeder/uml/X10_modtager_Class}
     \caption{Klassediagram for X10-udtag}
     \label{fig:X10_udtag_Class}
\end{figure}
