Her følger klassebeskrivelse som beskriver alle klasser, metoder og attributter i klasserne for X10-udtaget.

%
% X10IF
%
{\centering
\textbf{X10IF}\par
}
\textbf{Ansvar:} At varetage kommunikation mellem CSS hovedenhed og X10 udtag.

Dette er en global klasse da ISR for INT1 tilgår insertX10bit()-metoden i objektet. Det globale objekt hedder X10IFObj. \\
\textbf{Attributer:}
\begin{itemize}
	\item \textbf{char X10Array\_[ ]} \\
	Array til midlertidigt at holde modtaget X10 formateret kommando
	\item \textbf{char X10Kommando\_[ ]} \\
	Array til at holde X10 formateret kommando	
	\item \textbf{X10Adresse\_[ ]} \\
	Array til at holde X10 dekodet adresse
	\item \textbf{X10ArrayPlads\_} \\
	Index til næste plads i det midlertidige kommando array
	\item \textbf{X10KommandoPlads\_} \\
	Index til næste plads i kommando array
	\item \textbf{X10AdressePlads\_} \\
	Index til næste plads i adresse array
	\item \textbf{UC2Aktiver * UC2Ptr\_} \\
	Pointer til associeret UC2Aktiver objekt
	\item \textbf{UC3Deaktiver * UC3Ptr\_} \\
	Pointer til associeret UC3Deaktiver objekt
\end{itemize}
\textbf{Metoder:} \\
X10IF(); (Constructor) \\
\textbf{Parametre:} Ingen \\
\textbf{Returværdi:} Ingen \\
\textbf{Beskrivelse:} Initialiserer indgange og initialiserer alle members med 0 \\

void setPointer(UC2Aktiver * UC2Ptr, UC3Deaktiver * UC3Ptr); \\
\textbf{Parametre:} Pointer til associeret UC2Aktiver objekt, Poiner til associeret UC3Deaktiver objekt \\
\textbf{Returværdi:} Ingen \\
\textbf{Beskrivelse:} Initialiserer pointer members til associerede objekter \\

void aktiverINT1( ); \\
\textbf{Parametre:} Ingen \\
\textbf{Returværdi:} Ingen \\
\textbf{Beskrivelse:} Aktiverer INT1 ved toggle og aktiverer global interrupt \\

void deaktiverINT1( ); \\
\textbf{Parametre:} Ingen \\
\textbf{Returværdi:} Ingen \\
\textbf{Beskrivelse:} Deaktiver INT1 \\

void insertX10bit( char X10bit ); \\
\textbf{Parametre:} Gyldige værdier 0 og 1. Bemærk ikke ASCII værdier \\
\textbf{Returværdi:} Ingen \\
\textbf{Beskrivelse:} Indsætter X10bit i X10Array\_[], kontrollerer STX. Hvis STX er gyldigt hentes en fuld kommando ind og denne dekodes med metoden unwrapX10Array() \\

void unwrapX10Array( char * X10Array ); \\
\textbf{Parametre:} Pointer til fuld X10 formateret kommando \\
\textbf{Returværdi:} Ingen \\
\textbf{Beskrivelse:} Kontrollerer STX og ETX. Hvis disse er godkendt udledes kommando og adresse. Hvis kommando er aktiver eller deaktiver kaldes hhv. aktiver()-metoden i UC2Aktiver objektet eller deaktiver()-metoden i UC3Deaktiver. Her efter nulstilles index attributter. \\

ISR( INT1\_vect ); \\
\textbf{Parametre:} Adresse til INT1 vektor \\
\textbf{Returværdi:} Ingen \\
\textbf{Beskrivelse:} Deaktiverer interrupts. Venter et øjeblik og udlæser værdien på data indgangen. Gemme den læste data med insertX10bit()-metoden i det globale X10IF objekt. Aktiverer interrupt igen. \\

%
% UC2Aktiver
%
{\centering
\textbf{UC2Aktiver}\par
}
\textbf{Ansvar:} At varetage UC2 Aktiver forløbet \\
\textbf{Attributer:}
\begin{itemize}
	\item \textbf{AdresseIF * AIFPtr\_} \\
	Pointer til associeret AdresseIF objekt
	\item \textbf{UdtagIF * UIFPtr\_} \\
	Pointer til associeret UdtagIF objekt
\end{itemize}
UC2Aktiver( AdresseIF * AIFPtr, UdtagIF * UIFPtr ); (Constructor) \\
\textbf{Parametre:} Pointer til associeret AdresseIF objekt, Pointer til associeret UdtagIF objekt \\
\textbf{Returværdi:} Ingen \\
\textbf{Beskrivelse:} Initialiser pointerer til associerede objekter \\

\textbf{Metoder:} \\
void aktiver( char * adresse ); \\
\textbf{Parametre:} Pointer til array med adresse \\
\textbf{Returværdi:} Ingen \\
\textbf{Beskrivelse:} Kontroller adresse med adresseTjek()-metoden i AdresseIF objektet og aktiver udtag hvis de er identiske \\

%
% UC3Deaktiver
%
{\centering
\textbf{UC3Deaktiver}\par
}
\textbf{Ansvar:} At varetage UC3 Deaktiver forløbet \\
\textbf{Attributer:}
\begin{itemize}
	\item \textbf{AdresseIF * AIFPtr\_} \\
	Pointer til associeret AdresseIF objekt
	\item \textbf{UdtagIF * UIFPtr\_} \\
	Pointer til associeret UdtagIF objekt
\end{itemize}
\textbf{Metoder:} \\
UC3Deaktiver( AdresseIF * AIFPtr, UdtagIF * UIFPtr ); (Constructor) \\
\textbf{Parametre:} Pointer til associeret AdresseIF objekt, Pointer til associeret UdtagIF objekt \\
\textbf{Returværdi:} Ingen \\
\textbf{Beskrivelse:} Initialiser pointerer til associerede objekter \\

void deaktiver( char * adresse ); \\
\textbf{Parametre:} Pointer til array med adresse \\
\textbf{Returværdi:} Ingen \\
\textbf{Beskrivelse:} Kontroller adresse med adresseTjek()-metoden i AdresseIF objektet og deaktiver udtag hvis de er identiske \\

%
% AdresseIF
%
{\centering
\textbf{AdresseIf}\par
}
\textbf{Ansvar:} At kontrollerer modtaget adresse fra X10 interface og indstillet adresse på X10-udtag \\
\textbf{Attributer:}
\begin{itemize}
	\item \textbf{adresse\_[ ]} \\
	Array til X10-udtag adresse
\end{itemize}
\textbf{Metoder:} \\
AdresseIF( ); (Constructor) \\
\textbf{Parametre:} Ingen \\
\textbf{Returværdi:} Ingen \\
\textbf{Beskrivelse:} Indstil indgange og initialiser array med 0. \\

bool adresseTjek( char * adresse ); \\
\textbf{Parametre:} Pointer til array med adresse \\
\textbf{Returværdi:} True hvis adresser er ens, ellers false \\
\textbf{Beskrivelse:} Kontrollerer adresse og returnerer bool med status \\


%
% UdtagIF
%
{\centering
\textbf{UdtagIF}\par
}
\textbf{Ansvar:} At styre udtaget \\
\textbf{Attributer:} Ingen \\
\textbf{Metoder:} \\
UdtagIF( ); (Constructor) \\
\textbf{Parametre:} Ingen \\
\textbf{Returværdi:} Ingen \\
\textbf{Beskrivelse:} Indstil udgang \\

void aktiver( ); \\
\textbf{Parametre:} Ingen \\
\textbf{Returværdi:} Ingen \\
\textbf{Beskrivelse:} Sætter udgang til 1 \\

void deaktiver( ); \\
\textbf{Parametre:} Ingen \\
\textbf{Returværdi:} Ingen \\
\textbf{Beskrivelse:} Sætter udgang til 0 \\
