\chapter{Accepttestspecifikation}

\begin{table}[!htbp] \centering
\begin{tabular}{|p{2cm}|p{8cm}|}
	\hline
		\multicolumn{2}{|l|}{Versionshistorik} \\ \hline
		\textbf{v1.0} &24-03-2014 Hele gruppen (efter 1. review) \\ \hline
		\textbf{v0.5} &20-03-2014 Hele gruppen \\ \hline
	\end{tabular}
\end{table}

Punkterne i Accepttestspecifikationen, er skrevet ud fra punkterne i hovedforløbet, for de enkelte usecases.
Ved udførsel af accepttesten, bør UC\ref{ATUC7} laves først.

% Accepttest Use Case 1
%%% Tabel opsætning + Headernavn!!

\begin{center}
\begin{longtable}{|p{0,5cm}|p{3,8cm}|p{3,8cm}|p{2,2cm}|p{2,2cm}|} % l for left, c for center, r for right 
\hline
\multicolumn{5}{|l|}{\textbf{UC1: Login}} \\ \hline
\multicolumn{1}{|c|}{} &
\textbf{Test} &
\textbf{Forventet \newline Resultat} &
\textbf{Resultat} &
\textbf{Godkendt/ \newline Kommentar} \\ \hline 
\endfirsthead

\multicolumn{5}{l}{...fortsat fra forrige side} \\ \hline 
\multicolumn{1}{c|}{} &
\textbf{Test} &
\textbf{Forventet \newline Resultat} &
\textbf{Resultat} &
\textbf{Godkendt/ \newline Kommentar} \\ \hline 
\endhead

%%%% Tabel Opsætning

\textbf{1}		
&Bruger vælger login i interfacet
&Login skærm kommer frem på skærmen 	
&Som \newline forventet	
&Godkendt \\\hline
\textbf{2.1}		
&På DE2 board indtastes koderne ''0001'', ''0011'' og ''0111'' adskilt af tryk på KEY0 
&Skærm ændres til hovedmenuen
&Som \newline forventet	
&Godkendt \\\hline
\textbf{2.2}		
&På DE2 board indtastes koderne ''0001'', ''0011'' og ''0110'' adskilt af tryk på KEY0 
&Skærm forbliver på login siden	
&Som \newline forventet	
&Godkendt \\\hline
\textbf{2a}		
&Bruger vælger annuller	
&Skærm går til login siden	
&Som \newline forventet	
&Godkendt \\\hline

\textbf{3}		
&Systemet validerer adgangskoden		
&Indtastede adgangskode valideres af systemtet 	
&Som \newline forventet	
&Godkendt \\\hline
\textbf{3a}		
&Systemet nægter adgang og beder bruger om at indtaste adgangskode igen	
&Indtastede adgangskode ikke valideret af systemet. Der bedes igen om adgangskode	
&Som \newline forventet	
&Godkendt \\\hline
\textbf{4}		
&Bruger får adgang til hovedmenuen		
&Hovedmenuen vises
&Som \newline forventet	
&Godkendt \\\hline
	\end{longtable}
	%\caption{•}
	\label{ATUC1} 
\end{center}

% Accepttest Use Case 2
\begin{table}[htbp] \centering
\begin{tabular}{|p{1,7cm}|p{2,3cm}|p{3cm}|p{3cm}|p{3cm}|} % l for left, c for center, r for right 
	\hline
\multicolumn{5}{|l|}{\textbf{UC2: Deaktiver CSS enhed}} \\\hline
&\textbf{Test} &\textbf{Forventet \newline Resultat} &\textbf{Resultat} &\textbf{Godkendt/ \newline Kommentar} \\\hline
\textbf{Punkt 1}	&
Vælg "Deaktiver alt" &
Alle enheder måles, til at være deaktiveret &
N/A &
N/A \\\hline
\textbf{Punkt 2} &
Vælg Deaktiver alle låse &
Visuel: Se at låse bliver låst op &
N/A	&
N/A \\\hline
\textbf{Punkt 3} &
Deaktiver \newline babyalarm(er) &
Babyalarmen måles til at være slukket &
N/A &
N/A \\\hline
	\end{tabular}
	%\caption{•}
	\label{ATUC2} 
\end{table}

% Accepttest Use Case 3
\begin{table}[htbp] \centering
\begin{tabular}{|l|l|l|l|l|} % l for left, c for center, r for right 
	\hline
\multicolumn{5}{|l|}{\textbf{UC3: Udlæs status}} \\\hline
	&\textbf{Test} &\textbf{Forventet Resultat} &\textbf{Resultat} &\textbf{Godkendt/Kommentar} \\\hline
\textbf{Punkt 1}		&Test	&Test 	&Test	&Test \\\hline
\textbf{Punkt 2}		&Test	&Test 	&Test	&Test \\\hline
\textbf{Punkt 3}		&Test	&Test 	&Test	&Test \\\hline
\textbf{Punkt 4}		&Test	&Test 	&Test	&Test \\\hline
\textbf{Punkt 5}		&Test	&Test 	&Test	&Test \\\hline
\textbf{Punkt 6}		&Test	&Test 	&Test	&Test \\\hline
	\end{tabular}
	%\caption{•}
	\label{ATUC3} 
\end{table}

% Accepttest Use Case 4
\begin{table}[htbp] \centering
\begin{tabular}{|p{1,7cm}|p{2,3cm}|p{3cm}|p{3cm}|p{3cm}|} % l for left, c for center, r for right 
	\hline
\multicolumn{5}{|l|}{\textbf{UC4: Detekter brand}} \\\hline
&\textbf{Test} &\textbf{Forventet \newline Resultat} &\textbf{Resultat} &\textbf{Godkendt/ \newline Kommentar} \\\hline
\textbf{Punkt 1} &
Tilfør røg til sensor &
Se næste punkt &
Test & 
Test \\\hline
\textbf{Punkt 2} &
Alarm udløses &
Visuel/Auditiv: Alarmering &
Test	 &
Test \\\hline
\textbf{Punkt 3} &
Tryk på deaktiver knap &
Alarmering slukkes midlertidigt&
Test	 &
Test \\\hline
	\end{tabular}
	%\caption{•}
	\label{ATUC4} 
\end{table}

% Accepttest Use Case 5
%%% Tabel opsætning + Headernavn!!

\begin{center}
\begin{longtable}{|p{0,5cm}|p{3,8cm}|p{3,8cm}|p{2,2cm}|p{2,2cm}|} % l for left, c for center, r for right 
\hline
\multicolumn{5}{|l|}{\textbf{UC5: Detekter lyd}} \\ \hline
\multicolumn{1}{|c|}{} &
\textbf{Test} &
\textbf{Forventet \newline Resultat} &
\textbf{Resultat} &
\textbf{Godkendt/ \newline Kommentar} \\ \hline 
\endfirsthead

\multicolumn{5}{l}{...fortsat fra forrige side} \\ \hline 
\multicolumn{1}{|c|}{} &
\textbf{Test} &
\textbf{Forventet \newline Resultat} &
\textbf{Resultat} &
\textbf{Godkendt/ \newline Kommentar} \\ \hline 
\endhead

%%%% Tabel Opsætning


\textbf{1}		
&Lyddetektor er aktiveret
&Lyddetektor er aktiv	
&	
& \\\hline
\textbf{2}		
&Kontinuerligt lyd efterlignes	
&Detektorer opfanger lyd
&	
& \\\hline
\textbf{3}		
&Systemet underrettes	
&Systemet modtager signal fra lyddetektor 
&	
&  \\\hline
\textbf{4}		
&Systemet afsender SMS
&SMS-modtager modtager SMS fra systemet
& 	
& \\\hline
	\end{longtable}
	%\caption{•}
	\label{ATUC5} 
\end{center}

%\begin{table}[htbp] \centering
%\begin{tabular}{|p{1,7cm}|p{2,3cm}|p{3cm}|p{3cm}|p{3cm}|} % l for left, c for center, r for right 
%	\hline
%\multicolumn{5}{|l|}{\textbf{UC7: Advisering}} \\\hline
%&\textbf{Test} &\textbf{Forventet \newline Resultat} &\textbf{Resultat} &\textbf{Godkendt/ \newline Kommentar} \\\hline
%\textbf{Punkt 1}		&Advisering vælges i interface	&Advisering screen kommer frem på skærmen 	&N/A 	&N/A \\\hline
%\textbf{Punkt 2}		&Der indtastes ændringer	og bekræftes		&Oplysningerne lagers i systemet og brugeren bliver sendt tilbage til menuen 	&N/A 	&N/A \\\hline
%	\end{tabular}
%	%\caption{•}
%	\label{ATUC7} 
%\end{table}

% Accepttest Use Case 6
%%% Tabel opsætning + Headernavn!!

\begin{center}
\begin{longtable}{|p{0,5cm}|p{3,8cm}|p{3,8cm}|p{2,2cm}|p{2,2cm}|} % l for left, c for center, r for right 
\hline
\multicolumn{5}{|l|}{\textbf{UC6: Rediger SMS-modtager}} \\ \hline
\multicolumn{1}{|c|}{} &
\textbf{Test} &
\textbf{Forventet \newline Resultat} &
\textbf{Resultat} &
\textbf{Godkendt/ \newline Kommentar} \\ \hline 
\endfirsthead

\multicolumn{5}{l}{...fortsat fra forrige side} \\ \hline 
\multicolumn{1}{|c|}{} &
\textbf{Test} &
\textbf{Forventet \newline Resultat} &
\textbf{Resultat} &
\textbf{Godkendt/ \newline Kommentar} \\ \hline 
\endhead

%%%% Tabel Opsætning

\textbf{1} &
''Rediger SMS-modtager'' vælges i hovedmenu &
Menuen for ændring af SMS-bruger vises &
 &
 \\\hline

\textbf{2} &
Ændring fortages i SMS-modtagerens mobil nummer &
SMS-modtagerens mobil nummer opdateres i systemet &
 &
 \\\hline


	\end{longtable}
	%\caption{•}
	\label{ATUC6} 
\end{center}

% Accepttest Use Case 7
\begin{table}[htbp] \centering
\begin{tabular}{|l|l|l|l|l|} % l for left, c for center, r for right 
	\hline
\multicolumn{5}{|l|}{\textbf{UC7: Ændre SMS bruger}} \\\hline
	&\textbf{Test} &\textbf{Forventet Resultat} &\textbf{Resultat} &\textbf{Godkendt/Kommentar} \\\hline
\textbf{Punkt 1}		&JS	&Test 	&Test	&Test \\\hline
\textbf{Punkt 2}		&Test	&Test 	&Test	&Test \\\hline
\textbf{Punkt 3}		&Test	&Test 	&Test	&Test \\\hline
\textbf{Punkt 4}		&Test	&Test 	&Test	&Test \\\hline
\textbf{Punkt 5}		&Test	&Test 	&Test	&Test \\\hline
\textbf{Punkt 6}		&Test	&Test 	&Test	&Test \\\hline
	\end{tabular}
	%\caption{•}
	\label{ATUC7} 
\end{table}

% Accepttest Use Case 8
\begin{table}[htbp] \centering
\begin{tabular}{|p{1,7cm}|p{2,3cm}|p{3cm}|p{3cm}|p{3cm}|} % l for left, c for center, r for right 
	\hline
\multicolumn{5}{|l|}{\textbf{UC8: Login}} \\\hline
&\textbf{Test} &\textbf{Forventet \newline Resultat} &\textbf{Resultat} &\textbf{Godkendt/ \newline Kommentar} \\\hline
\textbf{Punkt 1}		
&Login vælges i interface	
&Login screen kommer frem på skærmen 	
&N/A 	
&N/A \\\hline
\textbf{Punkt 2}		
&Brugernavnet "Test Bruger"  oprettes og tildeles passwordet "test4321" Login forsøg foretages med disse parameter	
&brugernavn vises på skærmen, password karakter vises som "*" 	
&N/A 	
&N/A \\\hline
\textbf{Punkt 3}		
&Systemet validerer login information		
&Indtastede information vailders af systemtet 	
&N/A 	
&N/A \\\hline
\textbf{Punkt 4}		
&Bruger får adgang til hovedmenu		
&Hovedmenu vises på skærmen og er klar til brug 	
&N/A 	
&N/A  \\\hline
	\end{tabular}
	%\caption{•}
	\label{ATUC8} 
\end{table}

% Accepttest af ikke-funktionelle krav
%%% Tabel opsætning + Headernavn!!

\begin{center}
\begin{longtable}{|p{0,5cm}|p{3,8cm}|p{3,8cm}|p{2,2cm}|p{2,2cm}|} % l for left, c for center, r for right 
\hline
\multicolumn{5}{|l|}{\textbf{Ikke-funktionelle krav}} \\ \hline
\multicolumn{1}{|c|}{} &
\textbf{Test} &
\textbf{Forventet \newline Resultat} &
\textbf{Resultat} &
\textbf{Godkendt/ \newline Kommentar} \\ \hline 
\endfirsthead

\multicolumn{5}{l}{...fortsat fra forrige side} \\ \hline 
\multicolumn{1}{|c|}{} &
\textbf{Test} &
\textbf{Forventet \newline Resultat} &
\textbf{Resultat} &
\textbf{Godkendt/ \newline Kommentar} \\ \hline 
\endhead

%%%% Tabel Opsætning

\textbf{1} &
Udenforstående bruger gennemlæser manualen og opsætter systemet med et X10 udtag &
Brugeren har ikke problemer med opsætningen og brugen af systemet &
N/A &
N/A \\\hline

\textbf{2} &
Levetiden på 5 år er ikke testbart &
N/A &
N/A &
N/A \\\hline

\textbf{3} &
Software oppetid på 1 måned er ikke testbart &
N/A &
N/A &
N/A \\\hline

\textbf{4} &
Systemet antages som værende fuldt opsat.\newline
Bruger aktiverer et X10 udtag iht. UC2 Aktiver og kontrollerer tiden fra ''Aktiver'' er valgt til enhenden reagerer&
Tiden ligger inden for grænsen &
N/A &
N/A \\\hline

\textbf{5} &
Systemet antages som værende fuldt opsat.\newline
Der trykkes på Tænd/sluk knappen på hovedenheden og computeren. Når computeren er startet op, startes CSS programmet. &
Tiden ligger inden for grænsen &
N/A &
N/A \\\hline

\textbf{6} &
I testmiljøet produceres der ikke 15 X10 udtag og er derfor ikke testbart &
N/A  &
N/A &
N/A \\\hline

\textbf{7} &
Systemet antages som værende fuldt opsat.\newline
Lyddetektoren udsættes for et lydtryk ved at klappe kontinuert i 5 sekunder &
??? &
N/A &
N/A \\\hline

\textbf{8} &
Systemet antages som værende fuldt opsat.\newline
Et X10 udtag koblet op på systemet fjernes. Adressen aflæses og en ny enhed sættes i systemet med samme adresse.&
Det er muligt at kontrollere den nye enhed uden at ændre opsætning i systemet.&
N/A &
N/A \\\hline

\textbf{9} &
Systemet opsættes i et testmiljø som reflektere den almindelige bruger ved at udføre UC7&
At det ønskede X10 udtag kan styres. Heraf at systemet fungerer som forventet&
N/A &
N/A \\\hline

\textbf{10} &
Systemet antages som værende fuldt opsat.\newline
Et nyt X10 udtag opsættes ved at udføre UC8 &
X10 udtaget virker med systemet&
N/A &
N/A \\\hline

\textbf{11} &
Testet under punkt 9&
N/A &
N/A &
N/A \\\hline

\textbf{12} &
Systemet antages som værende fuldt opsat.\newline
??&
??&
N/A &
N/A \\\hline

\textbf{13} &
Testet under punkt 7&
N/A &
N/A &
N/A \\\hline

\textbf{14} &
Testes ikke på grund af begrænsninger i systemet, se sektion \ref{sectionBegraensninger}&
&
N/A &
N/A \\\hline

\textbf{15} &
Systemet antages som værende fuldt opsat.\newline
UC2 og UC3 udføres på et opsat X10 udtag&
Visueltest: En LED indikator viser at enhenden er aktiv&
N/A &
N/A \\\hline

\textbf{16} &
Testet under punkt 9&
N/A &
N/A &
N/A \\\hline

\textbf{17} &
Systemet antages som værende fuldt opsat.\newline
??&
??&
N/A &
N/A \\\hline

\textbf{18} &
Systemet antages som værende fuldt opsat.\newline
Lyddetektoren udsættes for lyd, i form af klap, to gange med 30 sekunders mellemrum.&
Der modtages kun 1 SMS-besked.&
N/A &
N/A \\\hline

\textbf{19} &
Testes ikke på grund af begrænsninger i systemet, se sektion \ref{sectionBegraensninger}&
N/A &
N/A &
N/A \\\hline

\textbf{20} &
Testes ikke på grund af begrænsninger i systemet, se sektion \ref{sectionBegraensninger}&
&
N/A &
N/A \\\hline



	\end{longtable}
	%\caption{•}
	\label{IKFUNK} 
\end{center}