\chapter{Accepttestspecifikation}

\begin{table}[!htbp] \centering
\begin{tabular}{|p{2cm}|p{8cm}|}
	\hline
		\multicolumn{2}{|l|}{Versionshistorik} \\ \hline
		\textbf{v1.0} &24-03-2014 Hele gruppen (efter 1. review) \\ \hline
		\textbf{v0.5} &20-03-2014 Hele gruppen \\ \hline
	\end{tabular}
\end{table}

Punkterne i Accepttestspecifikationen, er skrevet ud fra punkterne i hovedforløbet, for de enkelte usecases.
Ved udførsel af accepttesten, bør UC\ref{ATUC7} laves først.

% Accepttest Use Case 1
\begin{table}[htbp] \centering
	\begin{tabular}{|l|l|l|l|l|} % l for left, c for center, r for right 
		\hline
		\multicolumn{5} {|l|} {\textbf{UC1: Aktiver CSS enhed}} \\\hline
					&\textbf{Test} &\textbf{Forventet Resultat} 	&\textbf{Resultat} 	&\textbf{Godkendt/Kommentar} \\\hline
		
\textbf{Punkt 1}		&Der trykkes på knappen "Aktiver"	& Det er muligt at trykke på knappen				&N/A &N/A \\\hline
		
\textbf{Punkt 2}		&Visuel test: Billedet skifter til "Aktiver"-billedet og viser specificerede knapper																																& Brugerinterface viser mulige units samt specificerede knapper 																																						&N/A &N/A \\\hline
		
\textbf{Punkt 3}		&Bruger kan markerer ønskede units		& Det er muligt at markerer ønskede units 	&N/A &N/A \\\hline
		
\textbf{Punkt 4}		&Det trykkes på knappen "Aktiver"		& Det er muligt at trykke på knappen		 	&N/A &N/A \\\hline

\textbf{Punkt 4a}	&Der trykkes på knappen "Tilbage"		& Brugerinterface viser standardskærm
																										&N/A &N/A \\\hline

\textbf{Punkt 5}		&Det måles at valgte units bliver aktiveret				
															& De valgte units bliver aktiveret aktiveret	&N/A &N/A \\\hline
															
\textbf{Punkt 5a}	&Der vælges ingen units i punkt \ref{uc1select} og trykkes "Aktiver"				
															& Brugerinterface udskriver fejl på skærmen med besked om at vælge en unit og går til UC1.\ref{uc1select}																												&N/A &N/A \\\hline
		
\textbf{Punkt 6}		&Visuel test: Brugerinterface viser besked om at units er aktiverede
															&Brugerinterface viser besked 				&N/A &N/A \\\hline
\textbf{Punkt 7}		&Visuel test: Brugerinterface viser standardskærm
															&Brugerinterface viser standardskærm 		&N/A &N/A \\\hline
															
	\end{tabular}
	%\caption{•}
	\label{ATUC1} 
\end{table}

% Accepttest Use Case 2
\begin{table}[htbp] \centering
\begin{tabular}{|p{1,7cm}|p{2,3cm}|p{3cm}|p{3cm}|p{3cm}|} % l for left, c for center, r for right 
	\hline
\multicolumn{5}{|l|}{\textbf{UC2: Deaktiver CSS enhed}} \\\hline
&\textbf{Test} &\textbf{Forventet \newline Resultat} &\textbf{Resultat} &\textbf{Godkendt/ \newline Kommentar} \\\hline
\textbf{Punkt 1}	&
Vælg "Deaktiver alt" &
Alle enheder måles, til at være deaktiveret &
N/A &
N/A \\\hline
\textbf{Punkt 2} &
Vælg Deaktiver alle låse &
Visuel: Se at låse bliver låst op &
N/A	&
N/A \\\hline
\textbf{Punkt 3} &
Deaktiver \newline babyalarm(er) &
Babyalarmen måles til at være slukket &
N/A &
N/A \\\hline
	\end{tabular}
	%\caption{•}
	\label{ATUC2} 
\end{table}

% Accepttest Use Case 3
%%% Tabel opsætning

\begin{center}
\begin{longtable}{|p{1,8cm}|p{2,2cm}|p{3cm}|p{3cm}|p{3cm}|} % l for left, c for center, r for right 
\hline
\multicolumn{5}{|l|}{\textbf{UC3: Deaktiver}} \\ \hline
\multicolumn{1}{|c|}{} &
\textbf{Test} &
\textbf{Forventet \newline Resultat} &
\textbf{Resultat} &
\textbf{Godkendt/ \newline Kommentar} \\ \hline 
\endfirsthead

\multicolumn{5}{l}{...fortsat fra forrige side} \\ \hline 
\multicolumn{1}{|c|}{} &
\textbf{Test} &
\textbf{Forventet \newline Resultat} &
\textbf{Resultat} &
\textbf{Godkendt/ \newline Kommentar} \\ \hline 
\endhead

%%%% Tabel Opsætning

\textbf{Punkt 1}	&
Vælg "Deaktiver alt" &
Alle enheder måles, til at være deaktiveret &
N/A &
N/A \\\hline
\textbf{Punkt 2} &
Vælg Deaktiver alle låse &
Visuel: Se at låse bliver låst op &
N/A	&
N/A \\\hline
\textbf{Punkt 3} &
Deaktiver \newline babyalarm(er) &
Babyalarmen måles til at være slukket &
N/A &
N/A \\\hline
	\end{longtable}
	%\caption{•}
	\label{ATUC3} 
\end{center}

% Accepttest Use Case 4
\begin{table}[htbp] \centering
\begin{tabular}{|l|l|l|l|l|} % l for left, c for center, r for right 
	\hline
\multicolumn{5}{|l|}{\textbf{UC4: Detekter brand}} \\\hline
	&\textbf{Test} &\textbf{Forventet Resultat} &\textbf{Resultat} &\textbf{Godkendt/Kommentar} \\\hline
\textbf{Punkt 1}		&Test	&Test 	&Test	&Test \\\hline
\textbf{Punkt 2}		&Test	&Test 	&Test	&Test \\\hline
\textbf{Punkt 3}		&Test	&Test 	&Test	&Test \\\hline
\textbf{Punkt 4}		&Test	&Test 	&Test	&Test \\\hline
\textbf{Punkt 5}		&Test	&Test 	&Test	&Test \\\hline
\textbf{Punkt 6}		&Test	&Test 	&Test	&Test \\\hline
	\end{tabular}
	%\caption{•}
	\label{ATUC4} 
\end{table}

% Accepttest Use Case 5
%%% Tabel opsætning + Headernavn!!

\begin{center}
\begin{longtable}{|p{0,5cm}|p{3,8cm}|p{3,8cm}|p{2,2cm}|p{2,2cm}|} % l for left, c for center, r for right 
\hline
\multicolumn{5}{|l|}{\textbf{UC5: Detekter lyd}} \\ \hline
\multicolumn{1}{|c|}{} &
\textbf{Test} &
\textbf{Forventet \newline Resultat} &
\textbf{Resultat} &
\textbf{Godkendt/ \newline Kommentar} \\ \hline 
\endfirsthead

\multicolumn{5}{l}{...fortsat fra forrige side} \\ \hline 
\multicolumn{1}{|c|}{} &
\textbf{Test} &
\textbf{Forventet \newline Resultat} &
\textbf{Resultat} &
\textbf{Godkendt/ \newline Kommentar} \\ \hline 
\endhead

%%%% Tabel Opsætning


\textbf{1}		
&Lyddetektor er aktiveret
&Lyddetektor er aktiv	
&Ikke \newline testbart
&Ikke \newline godkendt \\\hline
\textbf{2}		
&Kontinuerligt lyd efterlignes	
&Detektorer opfanger lyd
&Ikke \newline testbart
&Ikke \newline godkendt \\\hline
\textbf{3}		
&Systemet underrettes	
&Systemet modtager signal fra lyddetektor 
&Som \newline forventet	
&Godkendt \newline vha. test kontakt  \\\hline
\textbf{4}		
&Systemet afsender SMS
&SMS-modtager modtager SMS fra systemet
&Som \newline forventet	
&Godkendt  \\\hline
	\end{longtable}
	%\caption{•}
	\label{ATUC5} 
\end{center}

%\begin{table}[htbp] \centering
%\begin{tabular}{|p{1,7cm}|p{2,3cm}|p{3cm}|p{3cm}|p{3cm}|} % l for left, c for center, r for right 
%	\hline
%\multicolumn{5}{|l|}{\textbf{UC7: Advisering}} \\\hline
%&\textbf{Test} &\textbf{Forventet \newline Resultat} &\textbf{Resultat} &\textbf{Godkendt/ \newline Kommentar} \\\hline
%\textbf{Punkt 1}		&Advisering vælges i interface	&Advisering screen kommer frem på skærmen 	&N/A 	&N/A \\\hline
%\textbf{Punkt 2}		&Der indtastes ændringer	og bekræftes		&Oplysningerne lagers i systemet og brugeren bliver sendt tilbage til menuen 	&N/A 	&N/A \\\hline
%	\end{tabular}
%	%\caption{•}
%	\label{ATUC7} 
%\end{table}

% Accepttest Use Case 6
\begin{table}[htbp] \centering
\begin{tabular}{|p{1,7cm}|p{2,3cm}|p{3cm}|p{3cm}|p{3cm}|} % l for left, c for center, r for right 
	\hline
\multicolumn{5}{|l|}{\textbf{UC6: Detekter klima}} \\\hline
&\textbf{Test} &\textbf{Forventet \newline Resultat} &\textbf{Resultat} &\textbf{Godkendt/ \newline Kommentar} \\\hline
\textbf{Punkt 1}		&Kontrolmåling og sammenligning med Punkt2 &Data er inden for tolerance værdier &
N/A	&
N/A \\\hline
\textbf{Punkt 2}		&Sendte informationer kontrolleres ift. kontrolmåling &
De er overens ift. tolerancer&
N/A	&
N/A \\\hline

\end{tabular}
%\caption{•}
\label{ATUC6} 
\end{table}

% Accepttest Use Case 7
\begin{table}[htbp] \centering
\begin{tabular}{|p{1,7cm}|p{2,3cm}|p{3cm}|p{3cm}|p{3cm}|} % l for left, c for center, r for right 
	\hline
\multicolumn{5}{|l|}{\textbf{UC7: Ændre SMS bruger}} \\\hline
&\textbf{Test} &\textbf{Forventet \newline Resultat} &\textbf{Resultat} &\textbf{Godkendt/ \newline Kommentar} \\\hline
\textbf{Punkt 1} &
Der vælges "Ændre SMS-bruger" &
Bliver videresendt til menuen for ændring af SMS-bruger &
N/A	&
N/A \\\hline
\textbf{Punkt 2}	 &
Ændringer bekræftes &
Oplysningerne lagers i systemet og brugeren bliver sendt tilbage til menuen &
N/A &
N/A \\\hline
	\end{tabular}
	%\caption{•}
	\label{ATUC7} 
\end{table}

% Accepttest Use Case 8
\begin{table}[htbp] \centering
\begin{tabular}{|p{1,7cm}|p{2,3cm}|p{3cm}|p{3cm}|p{3cm}|} % l for left, c for center, r for right 
	\hline
\multicolumn{5}{|l|}{\textbf{UC8: Login}} \\\hline
&\textbf{Test} &\textbf{Forventet \newline Resultat} &\textbf{Resultat} &\textbf{Godkendt/ \newline Kommentar} \\\hline
\textbf{Punkt 1}		&Login vælges i interface	&Login screen kommer frem på skærmen 	&N/A 	&N/A \\\hline
\textbf{Punkt 2}		&Der indtastes brugernavn og password	&brugernavn vises på skærmen, password karakter vises som "*" 	&N/A 	&N/A \\\hline
\textbf{Punkt 3}		&Systemt vailderer login information		&Indtastede information vailders af systemtet 	&N/A 	&N/A \\\hline
\textbf{Punkt 4}		&Bruger får adgang til hovedmenu		&Hovedmenu vises på skærmen og er klar til brug 	&N/A 	&N/A  \\\hline
	\end{tabular}
	%\caption{•}
	\label{ATUC7} 
\end{table}

% Accepttest af ikke-funktionelle krav
%%% Tabel opsætning

\begin{center}
\begin{longtable}{|p{1,8cm}|p{2,2cm}|p{3cm}|p{3cm}|p{3cm}|} % l for left, c for center, r for right 
\hline
\multicolumn{5}{|l|}{\textbf{Ikke-funktionelle krav}} \\ \hline
\textbf{Punkt} &
\textbf{Test} &
\textbf{Forventet \newline Resultat} &
\textbf{Resultat} &
\textbf{Godkendt/ \newline Kommentar} \\ \hline 
\endfirsthead

\multicolumn{5}{l}{...fortsat fra forrige side} \\ \hline 
\textbf{Punkt} &
\textbf{Test} &
\textbf{Forventet \newline Resultat} &
\textbf{Resultat} &
\textbf{Godkendt/ \newline Kommentar} \\ \hline 
\endhead

%%%% Tabel Opsætning

\textbf{1} &
Udenforstående bruger gennemlæser manualen og opsætter systemet med et X10 udtag &
Brugeren har ikke problemer med opsætningen og brugen af systemet &
N/A &
N/A \\\hline

\textbf{2} &
Ikke testbart &
N/A &
N/A &
N/A \\\hline

\textbf{3} &
Ikke testbart &
N/A &
N/A &
N/A \\\hline

\textbf{4} &
Systemet antages som værende fuldt opsat.\newline
Bruger aktiverer et X10 udtag iht. UC2 Aktiver og kontrollerer tiden fra ''Aktiver'' er valgt til enhenden reagerer&
Tiden ligger inden for grænsen &
N/A &
N/A \\\hline

\textbf{5} &
Systemet antages som værende fuldt opsat.\newline
Der trykkes på Tænd/sluk knappen på hovedenheden og computeren. Når computeren er startet op startes CSS programmet. &
Tiden ligger inden for grænsen &
N/A &
N/A \\\hline

\textbf{6} &
Ikke testbart &
N/A  &
N/A &
N/A \\\hline

\textbf{7} &
Systemet antages som værende fuldt opsat.\newline
Babyalarmen udsættes for et lydtryk ved at klappe kontinuert i 5 sekunder &
??? &
N/A &
N/A \\\hline

\textbf{8} &
Systemet antages som værende fuldt opsat.\newline
Et X10 udtag koblet op på systemet fjernes. Adressen aflæses og en ny enhed sættes i systemet med samme adresse.&
Det er muligt at kontrollere den nye enhed uden at ændre opsætning i systemet.&
N/A &
N/A \\\hline

\textbf{9} &
Systemet opsættes i et testmiljø som reflektere den almindelige bruger ved at udføre UC7&
Systemet fungerer&
N/A &
N/A \\\hline

\textbf{10} &
Systemet antages som værende fuldt opsat.\newline
Et nyt X10 udtag opsættes ved at udføre UC8 &
X10 udtaget virker med systemet&
N/A &
N/A \\\hline

\textbf{11} &
Testet under punkt 9&
N/A &
N/A &
N/A \\\hline

\textbf{12} &
Systemet antages som værende fuldt opsat.\newline
??&
??&
N/A &
N/A \\\hline

\textbf{13} &
Testet under punkt 7&
N/A &
N/A &
N/A \\\hline

\textbf{14} &
Testes ikke på grund af begrænsninger i systemet, se sektion \ref{sectionBegraensninger}&
&
N/A &
N/A \\\hline

\textbf{15} &
Systemet antages som værende fuldt opsat.\newline
UC2 og UC3 udføres på et opsat X10 udtag&
Visueltest: En LED indikator viser at enhenden er aktiv&
N/A &
N/A \\\hline

\textbf{16} &
Testet under punkt punkt 9&
N/A &
N/A &
N/A \\\hline

\textbf{17} &
Systemet antages som værende fuldt opsat.\newline
??&
??&
N/A &
N/A \\\hline

\textbf{18} &
Systemet antages som værende fuldt opsat.\newline
Lyddetektoren udsættes for lyd, i form af klap, to gange med 30 sekunders mellemrum.&
Der modtages kun 1 SMS-besked.&
N/A &
N/A \\\hline

\textbf{19} &
Testes ikke på grund af begrænsninger i systemet, se sektion \ref{sectionBegraensninger}&
N/A &
N/A &
N/A \\\hline

\textbf{20} &
Testes ikke på grund af begrænsninger i systemet, se sektion \ref{sectionBegraensninger}&
&
N/A &
N/A \\\hline



	\end{longtable}
	%\caption{•}
	\label{ATUC8} 
\end{center}