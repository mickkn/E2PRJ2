%%% Tabel opsætning + Headernavn!!

\begin{center}
\begin{longtable}{|p{0,5cm}|p{3,8cm}|p{3,8cm}|p{2,2cm}|p{2,2cm}|} % l for left, c for center, r for right 
\hline
\multicolumn{5}{|l|}{\textbf{Ikke-funktionelle krav}} \\ \hline
\multicolumn{1}{|c|}{} &
\textbf{Test} &
\textbf{Forventet \newline Resultat} &
\textbf{Resultat} &
\textbf{Godkendt/ \newline Kommentar} \\ \hline 
\endfirsthead

\multicolumn{5}{l}{...fortsat fra forrige side} \\ \hline 
\multicolumn{1}{|c|}{} &
\textbf{Test} &
\textbf{Forventet \newline Resultat} &
\textbf{Resultat} &
\textbf{Godkendt/ \newline Kommentar} \\ \hline 
\endhead

%%%% Tabel Opsætning

\textbf{1} &
Udenforstående bruger gennemlæser manualen og opsætter systemet med et X10 udtag &
Brugeren har ikke problemer med opsætningen og brugen af systemet &
 &
  \\\hline

\textbf{2} &
Levetiden på 5 år er ikke testbart &
N/A &
 &
 \\\hline

\textbf{3} &
Software oppetid på 1 måned er ikke testbart &
N/A &
 &
 \\\hline

\textbf{4} &
Systemet antages som værende fuldt opsat.\newline
Bruger aktiverer et X10 udtag iht. UC2 Aktiver og kontrollerer tiden fra ''Aktiver'' er valgt til enhenden reagerer&
Tiden ligger inden for grænsen &
 &
 \\\hline

\textbf{5} &
Systemet antages som værende fuldt opsat.\newline
Der trykkes på Tænd/sluk knappen på hovedenheden og computeren. Når computeren er startet op, startes CSS programmet. &
Tiden ligger inden for grænsen &
 &
 \\\hline

\textbf{6} &
I testmiljøet produceres der ikke 15 X10 udtag og er derfor ikke testbart &
N/A  &
 &
 \\\hline

\textbf{7} &
Systemet antages som værende fuldt opsat.\newline
Lyddetektoren udsættes for et lydtryk ved at klappe kontinuert i 5 sekunder &
??? &
 &
 \\\hline

\textbf{8} &
Systemet antages som værende fuldt opsat.\newline
Et X10 udtag koblet op på systemet fjernes. Adressen aflæses og en ny enhed sættes i systemet med samme adresse.&
Det er muligt at kontrollere den nye enhed uden at ændre opsætning i systemet.&
 &
 \\\hline

\textbf{9} &
Systemet opsættes i et testmiljø som reflektere den almindelige bruger ved at udføre UC7&
At det ønskede X10 udtag kan styres. Heraf at systemet fungerer som forventet&
 &
 \\\hline

\textbf{10} &
Systemet antages som værende fuldt opsat.\newline
Et nyt X10 udtag opsættes ved at udføre UC8 &
X10 udtaget virker med systemet&
 &
 \\\hline

\textbf{11} &
Testet under punkt 9&
N/A &
N/A &
N/A \\\hline

\textbf{12} &
Systemet antages som værende fuldt opsat.\newline
??&
??&
 &
 \\\hline

\textbf{13} &
Testet under punkt 7&
N/A &
 &
 \\\hline

\textbf{14} &
Testes ikke på grund af begrænsninger i systemet, se sektion \ref{sectionBegraensninger}&
&
 &
 \\\hline

\textbf{15} &
Systemet antages som værende fuldt opsat.\newline
UC2 og UC3 udføres på et opsat X10 udtag&
Visueltest: En LED indikator viser at enhenden er aktiv&
 &
 \\\hline

\textbf{16} &
Testet under punkt 9&
N/A &
 &
 \\\hline

\textbf{17} &
Systemet antages som værende fuldt opsat.\newline
??&
??&
 &
 \\\hline

\textbf{18} &
Systemet antages som værende fuldt opsat.\newline
Lyddetektoren udsættes for lyd, i form af klap, to gange med 30 sekunders mellemrum.&
Der modtages kun 1 SMS-besked.&
 &
 \\\hline

\textbf{19} &
Testes ikke på grund af begrænsninger i systemet, se sektion \ref{sectionBegraensninger}&
N/A &
 &
 \\\hline

\textbf{20} &
Testes ikke på grund af begrænsninger i systemet, se sektion \ref{sectionBegraensninger}&
&
 &
 \\\hline



	\end{longtable}
	%\caption{•}
	\label{IKFUNK} 
\end{center}