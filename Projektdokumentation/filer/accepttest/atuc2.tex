%%% Tabel opsætning

\begin{center}
\begin{longtable}{|p{1,8cm}|p{2,2cm}|p{3cm}|p{3cm}|p{3cm}|} % l for left, c for center, r for right 
\hline
\multicolumn{5}{|l|}{\textbf{UC2: Aktiver}} \\ \hline
\multicolumn{1}{|c|}{} &
\textbf{Test} &
\textbf{Forventet \newline Resultat} &
\textbf{Resultat} &
\textbf{Godkendt/ \newline Kommentar} \\ \hline 
\endfirsthead

\multicolumn{5}{l}{...fortsat fra forrige side} \\ \hline 
\multicolumn{1}{|c|}{} &
\textbf{Test} &
\textbf{Forventet \newline Resultat} &
\textbf{Resultat} &
\textbf{Godkendt/ \newline Kommentar} \\ \hline 
\endhead

%%%% Tabel Opsætning

\textbf{Punkt 1}		&Bruger logger ind med kode	
					&\multicolumn{3}{l|}{Testes i UC1: Login} \\\hline
		
\textbf{Punkt 2}		&Bruger vælger ''Aktiver'' i hovedmenu																
					&UI fortsætter til Punkt 3 (''Aktiver menu'')
					&N/A 
					&N/A \\\hline
					
\textbf{Punkt 3}		&Visuel test: Visning af ''Aktiver menu''																
					&UI viser ''Aktiver menu''
					&N/A 
					&N/A \\\hline
		
\textbf{Punkt 4a}	&''Vælg alle'' vælges		
					&Alle enheder markeres på skærmen		 	
					&N/A 
					&N/A \\\hline

\textbf{Punkt 4b}	&Enkelte enheder vælges
					&De valgte ''enkelte'' enheder markeres på skærmen
					&N/A 
					&N/A \\\hline

\textbf{Punkt 5}		&''Aktiver'' vælges			
					&UI fortsætter til Punkt 6 (Aktivering)
					&N/A 
					&N/A \\\hline
					
\textbf{Punkt 5a}	&''Tilbage'' vælges			
					&Fortsætter til Punkt 8 (Viser hovedmenu)
					&N/A 
					&N/A \\\hline

\textbf{Punkt 6}		&Aktivering			
					&Valgte enheder måles aktiveret
					&N/A 
					&N/A \\\hline
															
\textbf{Punkt 6a}	&Der vælges ingen enheder og trykkes "Aktiver"				
					&UI udskriver fejl på skærmen med besked om at vælge en enhed og går til UC2.\ref{uc2menu}	. 
					 Der måles ingen ændringer på enhederne
					&N/A 
					&N/A \\\hline
		
\textbf{Punkt 7}		&Visuel test: Viser besked om at enheder er aktiverede
					&UI viser besked
					&N/A
					&N/A \\\hline
					
\textbf{Punkt 8}		&Visuel test: Viser hovedmenu
					&UI viser hovedmenu
					&N/A
					&N/A \\\hline

%\textbf{Version}		&\multicolumn{4}{l|}{1.1} \\ \hline
															
	\end{longtable}
	%\caption{•}
	\label{ATUC2} 
\end{center}