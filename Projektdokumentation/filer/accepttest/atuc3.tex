%%% Tabel opsætning + Headernavn!!

\begin{center}
\begin{longtable}{|p{0,5cm}|p{3,8cm}|p{3,8cm}|p{2,2cm}|p{2,2cm}|} % l for left, c for center, r for right 
\hline
\multicolumn{5}{|l|}{\textbf{UC3: Deaktiver}} \\ \hline
\multicolumn{1}{|c|}{} &
\textbf{Test} &
\textbf{Forventet \newline Resultat} &
\textbf{Resultat} &
\textbf{Godkendt/ \newline Kommentar} \\ \hline 
\endfirsthead

\multicolumn{5}{l}{...fortsat fra forrige side} \\ \hline 
\multicolumn{1}{|c|}{} &
\textbf{Test} &
\textbf{Forventet \newline Resultat} &
\textbf{Resultat} &
\textbf{Godkendt/ \newline Kommentar} \\ \hline 
\endhead

%%%% Tabel Opsætning

\textbf{1}			&Bruger logger ind med kode	
					&\multicolumn{3}{l|}{Testes i UC1: Login} \\\hline
		
\textbf{2}			&Bruger vælger ''Deaktiver'' i hovedmenu																
					&UI fortsætter til Punkt 3 (''Deaktiver menu'')
					&N/A 
					&N/A \\\hline
					
\textbf{3}			&Visuel test: Visning af ''Deaktiver menu''																
					&UI viser ''Deaktiver menu''
					&N/A 
					&N/A \\\hline
		
\textbf{4a}			&''Vælg alle'' vælges		
					&Alle enheder markeres på skærmen		 	
					&N/A 
					&N/A \\\hline

\textbf{4b}			&Enkelte enheder vælges
					&De valgte ''enkelte'' enheder markeres på skærmen
					&N/A 
					&N/A \\\hline

\textbf{5}			&''Deaktiver'' vælges			
					&UI fortsætter til Punkt 6 (Deaktivering)
					&N/A 
					&N/A \\\hline
					
\textbf{5a}			&''Tilbage'' vælges			
					&Fortsætter til Punkt 8 (Viser hovedmenu)
					&N/A 
					&N/A \\\hline

\textbf{6}			&Deaktivering			
					&Valgte enheder måles deaktiveret
					&N/A 
					&N/A \\\hline
															
\textbf{6a}			&Der vælges ingen enheder og trykkes "Deaktiver"				
					&UI udskriver fejl på skærmen med besked om at vælge en enhed og går til UC2.\ref{uc3menu}	. 
					 Der måles ingen ændringer på enhederne
					&N/A 
					&N/A \\\hline
		
\textbf{7}			&Visuel test: Viser besked om at enheder er deaktiverede
					&UI viser besked
					&N/A
					&N/A \\\hline
					
\textbf{8}			&Visuel test: Viser hovedmenu
					&UI viser hovedmenu
					&N/A
					&N/A \\\hline
					
%\textbf{Version}		&\multicolumn{4}{l|}{1.1} \\ \hline

	\end{longtable}
	%\caption{•}
	\label{ATUC3} 
\end{center}