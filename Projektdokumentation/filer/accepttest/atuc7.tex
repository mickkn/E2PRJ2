%%% Tabel opsætning

\begin{center}
\begin{longtable}{|p{1,8cm}|p{2,2cm}|p{3cm}|p{3cm}|p{3cm}|} % l for left, c for center, r for right 
\hline
\multicolumn{5}{|l|}{\textbf{UC7: Startopsætning}} \\ \hline
\multicolumn{1}{|c|}{} &
\textbf{Test} &
\textbf{Forventet \newline Resultat} &
\textbf{Resultat} &
\textbf{Godkendt/ \newline Kommentar} \\ \hline 
\endfirsthead

\multicolumn{5}{l}{...fortsat fra forrige side} \\ \hline 
\textbf{Punkt} &
\textbf{Test} &
\textbf{Forventet \newline Resultat} &
\textbf{Resultat} &
\textbf{Godkendt/ \newline Kommentar} \\ \hline 
\endhead

%%%% Tabel Opsætning

\textbf{1} &
Indsæt serielt kommunikationskabel (RS232) i mellem computer og hovedenhedens COM-port\newline
Indsæt styrekabel mellem babyalarm og hovedenheden\newline
Indsæt strømkabel mellem ledigt 230 Vac udtag og hovedenhedens AC indgang &
Visueltest: Alle kabler er forbundet korrekt &
N/A &
N/A \\\hline

\textbf{2} &
Tænd hovedenhed og computer &
Visueltest: Systemet starter op inden for kravet på maksimalt 2 minutter &
N/A &
N/A \\\hline

\textbf{3} &
Start CSS programmet på computeren &
Visueltest: Programmet starter op og viser hovedskærmen &
N/A &
N/A \\\hline

\textbf{4} &
En enhed opsættes ved at udføre accepttest af UC8 &
Den opsatte enhed er opsat korrekt &
N/A &
N/A \\\hline

\textbf{\rep{uc7sms}} &
SMS-modtager ændres ved at udføre accepttest af UC6 &
SMS-modtager er ændret &
N/A &
N/A \\\hline

	\end{longtable}
	%\caption{•}
	\label{ATUC7} 
\end{center}