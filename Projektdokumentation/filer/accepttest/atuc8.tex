%%% Tabel opsætning

\begin{center}
\begin{longtable}{|p{1,8cm}|p{2,2cm}|p{3cm}|p{3cm}|p{3cm}|} % l for left, c for center, r for right 
\hline
\multicolumn{5}{|l|}{\textbf{UC8: Tilføj/fjern X10 udtag}} \\ \hline
\textbf{Punkt} &
\textbf{Test} &
\textbf{Forventet \newline Resultat} &
\textbf{Resultat} &
\textbf{Godkendt/ \newline Kommentar} \\ \hline 
\endfirsthead

\multicolumn{5}{l}{...fortsat fra forrige side} \\ \hline 
\textbf{Punkt} &
\textbf{Test} &
\textbf{Forventet \newline Resultat} &
\textbf{Resultat} &
\textbf{Godkendt/ \newline Kommentar} \\ \hline 
\endhead

%%%% Tabel Opsætning

\textbf{1} &
Se accepttest af UC1 &
Bruger er logget ind og kan se hovedskærmen &
N/A &
N/A \\\hline

\textbf{2} &
Indstil X10 udtagets adresseswitch til adressen ''0101'' (1234) &
Visueltest: Adressen er indstillet korrekt &
N/A &
N/A \\\hline

\textbf{3} &
Vælg menupunkt ''Tilføj/fjern X10 udtag'' &
Visueltest: Programmet udskriver beskeden ''Indtast den fire cifrede adresse'' &
N/A &
N/A \\\hline

\textbf{4} &
Indtast adressen ''0101'' og tryk på ''enter'' knappen &
?? &
N/A &
N/A \\\hline

\textbf{4a} &
Indtast adressen ''0000'' og tryk på ''enter'' knappen &
Programmet udskriver fejlbeskeden og går til UC8.2 &
N/A &
N/A \\\hline

\textbf{4b} &
Indtast adressen ''0'' og tryk på ''enter'' knappen &
Visueltest: Programmet udskriver fejlbeskeden og går til UC8.2 &
N/A &
N/A \\\hline

\textbf{5} &
N/A &
Visueltest: Programmet udskriver beskeden ''Indtast navn''&
N/A &
N/A \\\hline

\textbf{6} &
Indtast ''Test enhed'' og tryk på ''enter'' knappen &
??  &
N/A &
N/A \\\hline

\textbf{6a} &
Indtast ''A'' og tryk på ''enter'' knappen.\newline
Gentag med ''abcdefghijklmnopqrstuvxwabcdefghijklmnopqrstuvxwab &
Visueltest: Programmet udskriver fejlbeskeden og går til UC8.2 &
N/A &
N/A \\\hline

\textbf{7} &
N/A &
Visueltest: Programmet returnerer til hovedskærmen &
N/A &
N/A \\\hline

\textbf{8} &
Indsæt X10 udtag i et 230 Vac udtag som er forbundet til systemet og kør accepttest af UC2 på den nyopsatte enhed&
Det er muligt at styre det opsatte X10 udtag ved brug af UC2 og UC3&
N/A &
N/A \\\hline


	\end{longtable}
	%\caption{•}
	\label{ATUC8} 
\end{center}