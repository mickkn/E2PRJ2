\chapter{Kravspecifikation}

\begin{table}[!htbp] \centering
\begin{tabular}{|p{2cm}|p{8cm}|}
	\hline
		\multicolumn{2}{|l|}{Versionshistorik} \\\hline
		\textbf{v1.1} &19-05-2014 Indledning + DE2 som aktør\\\hline
		\textbf{v1.0} &24-03-2014 Hele gruppen (efter 1. review)\\\hline
		\textbf{v0.5} &20-03-2014 Hele gruppen\\\hline
	\end{tabular}
\end{table}

\chapter{Indledning (BS, JS)}

Med udgangspunkt i børnesikkerhed i hjemmet er der blevet udviklet et produkt, som kan hjælpe familier med børn, til at få et mere sikkert hjem.

Af problemstillinger som kan opstå i en almindelig husholdning kan nævnes:
\begin{itemize}
	\item Fare for at et barn tænder for en kogeplade, eller andre elektriske varme aggregater, og efterfølgende kan brænde sig
	\item Fare for at et barn kan skære sig på køkkenknive som ligger i en skuffe
\end{itemize}

Den anden del af systemet er en babyalarm. Næsten alle mennesker i Danmark har deres mobiltelefon i nærheden hele tiden, så i stedet for at skulle have en babyalarm med rundt også, så kan man koble sin mobil til systemet og få besked når barnet giver lyd fra sig.

Dette ender ud i tre produkter:

\begin{itemize}
\item Afbryder til valgt 230 Vac stikkontakt
\subitem Beskyttelse mod kogeplader og lignende
\item Låsemekanisme til at låse skabe og skuffer
\subitem Aflåsning af skuffe med køkkenknive
\item Babyalarm til lyddetektering
\subitem SMS-beskeder i stedet for en ekstra ''boks'' i lommen
\end{itemize}

Systemet skal være nemt at sætte op og skal kommunikere over det eksisterende 230 V vekselspændings netværk i hus installationen.

En central computer håndterer styringen mellem enhederne og systemet kan aktiveres med et kodetryk.

For at udarbejde det endelige produkt er vi startet i fællesskab med at få lavet de nødvendige dokumentationer der skal bruges for at kunne udvikle sådan et produkt. Herunder er bl.a. use case diagram, kravspecifikation og accepttest. Herefter har gruppen delt sig i et hardware og software team. Hardware har taget til opgave at få udviklet de nødvendige elektroniske komponenter mens software har skrevet koden.

Begge teams har stået for den respektive dokumentation for det de fik udviklet gennem forløbet. 

\begin{figure}[htbp] \centering
\section{Aktører}
\fbox{\includegraphics[width=0.9\textwidth]{billeder/diagrammer/Kontekst_Diagram}}
\caption{Kontekst diagram}
\label{lab:kontekstdiagram}
\end{figure}

\begin{table}[!htbp] \centering
\subsection{Bruger}
\begin{tabular}{|p{4cm}|p{8cm}|}
	\hline
		\textbf{Type Beskrivelse} &
			Bruger aktøren er ejeren af systemet eller den voksne med adgang til Computeren. 
			Vil typisk være forældre, barnepige osv. (Primær) \\\hline
	\end{tabular}
\end{table}

\begin{table}[!htbp] \centering
\subsection{Eksterne enheder}
\begin{tabular}{|p{4cm}|p{8cm}|}
	\hline
		\textbf{Type Beskrivelse} &
			Eksterne enheder, omfatter hvad man ønsker at aflåse eller slukke for. 
			Vil typisk være skabe, komfur, el-kedel osv. (Sekundær) \\\hline
	\end{tabular}
\end{table}

\begin{table}[!htbp] \centering
\subsection{Barn}
\begin{tabular}{|p{4cm}|p{8cm}|}
	\hline
		\textbf{Type Beskrivelse} &
			Barnet eller børnene i huset, som systemet skal beskytte.	(Sekundær) \\\hline
	\end{tabular}
\end{table}

\begin{table}[!htbp] \centering
\subsection{SMS modtager}
\begin{tabular}{|p{4cm}|p{8cm}|}
	\hline
		\textbf{Type Beskrivelse} &
			Typisk forældrene eller barnepigen. Den person der skal have besked om gråd eller anden støj fra børneværelset. (Sekundær) \\\hline
	\end{tabular}
\end{table}

\begin{table}[!htbp] \centering
\subsection{DE2 Board}
\begin{tabular}{|p{4cm}|p{8cm}|}
	\hline
		\textbf{Type Beskrivelse} &
			DE2 Board programmeret som kodelås i DSD øvelse 7 (Sekundær) \\\hline
	\end{tabular}
\end{table}

\begin{figure}[!htbp] \centering
\section{Usecases}
\vspace*{\fill}
\includegraphics[width=\textwidth]{billeder/diagrammer/Usecase_Diagram}
\caption{Usecase diagram}
\label{lab:usecasediagram}
\vspace*{\fill}
\end{figure}

% UC1: Login

\subsection{UC1: Login}
\begin{table}[H] \centering
\begin{tabular}{|p{6cm}|p{8cm}|}
	\hline
<<<<<<< HEAD
\textbf{Mål}								
&At tilmeldt bruger af systemet kan logge ind ved brug af personlig brugernavn og password
 \\\hline
\textbf{Initialisering}					
&Bruger vælger login i interface
 \\\hline
\textbf{Aktører og Stakeholders}			
&Primær: Bruger
 \\\hline
\textbf{Referencer}						
&Ingen
 \\\hline
\textbf{Antal af samtidige hændelser}	
&Der kan fortages ét login ad gangen (sådan skal det formuleres!)
 \\\hline
\textbf{Forudsætning}					
&At interface er online
 \\\hline
\textbf{Efterfølgende tilstand}			
&At bruger er logget ind og hovedmenu vises på skærmen. Hele systmet er klar til brug
 \\\hline
\textbf{Hovedforløb}						
& 
\begin{enumerate}

\item Bruger vælger login i interface

\item \label{UC8und1}Bruger indtaster personlig brugernavn og adgangskode [Undtagelse 1: Bruger vælger Annuller]

\item \label{UC8und2} Systemet validerer brugernavn og adgangskode [Undtagelse 2: Ikke valideret]

\item Bruger får adgang til hovedmenu
 
\end{enumerate}
\\\hline

\textbf{Undtagelser}						
&\begin{enumerate}[label= \ref{UC8und1}a.]
			\item Bruger vælger annuller og kommer tilbage til startskærm
		\end{enumerate}
											
		\begin{enumerate}[label= \ref{UC8und2}a.]
			\item Brugernavn eller adgangskode ikke indtastet korret. Brugernavn og adganskode indtastes igen.
=======
		\multicolumn{2}{|l|}{\textbf{UC1: Aktiver CSS enhed(er)}} \\\hline
		
		\textbf{Mål}							&At brugeren kan aktivere enkelte eller alle enheder, i systemet.	\\\hline
		\textbf{Initialisering}				&Bruger vælger "Aktiver". 										\\\hline
		\textbf{Aktører og Stakeholders}		&Primær: Bruger ønsker at aktivere CSS enheder					\\\hline
		\textbf{Referencer}					&Login															\\\hline
		\textbf{Antal af samtidige hændelser}&1 																\\\hline
		\textbf{Forudsætning}				&Ingen															\\\hline
		\textbf{Efterfølgende tilstand}		&Hovedmenu vises 												\\\hline
		\textbf{Hovedforløb}					
			&\begin{enumerate}
	
				\item Bruger trykker på "Aktiver" knap
				
				\item Bruger logger ind med kode.
										
				\item Interface viser mulige enheder samt "Vælg alle", "Aktiver" og "Tilbage"-knapper
												
				\item \label{uc1select} Bruger markerer ønskede enheder til aktivering
												
				\item \label{uc1ex1} Bruger trykker "Aktiver"\newline
					\textbf{[Undtagelse \ref{uc1ex1}a]} Bruger trykker "Tilbage"
												
				\item \label{uc1ex2} Systemet aktiverer valgte enheder\newline
					\textbf{[Undtagelse \ref{uc1ex2}a]} Ingen valgte enheder
				
				\item Brugerinterface viser besked om at enheder, er aktiverede
																	
				\item Interface returnerer til hovedmenu
												
			\end{enumerate}\\\hline
		
		\textbf{Undtagelser}					
		&\begin{enumerate}[label= \ref{uc1ex1}a.]
			\item Brugerinterface returnerer til standardskærm og UC1 afbrydes
		\end{enumerate}
											
		\begin{enumerate}[label= \ref{uc1ex2}a.]
			\item Hvis ingen unit er valgt udskrives en fejl på skærmen og beder brugeren om at vælge en enhed og går til UC1.\ref{uc1select}.
>>>>>>> FETCH_HEAD
		\end{enumerate} \\\hline


		\textbf{Version}		&1.0 \\\hline
	\end{tabular}
	\label{UC1} 
\end{table}

% UC2: Aktiver

\subsection{UC2: Aktiver}
\begin{table}[H] \centering
	\begin{tabular} {|p{6cm}|p{8cm}|}
	\hline		
		\textbf{Mål}							&At Bruger kan aktivere enkelte eller alle enheder, i systemet\\\hline
		\textbf{Initialisering}				&Bruger vælger "Aktiver" 	\\\hline
		\textbf{Aktører og Stakeholders}		&Bruger(Primær) 				\\\hline
		\textbf{Referencer}					&UC1: Login					\\\hline
		\textbf{Antal af samtidige hændelser}&1 							\\\hline
		\textbf{Forudsætning}				&Bruger er logget ind i systemet\\\hline
		\textbf{Efterfølgende tilstand}		&Enkelte eller alle enheder er aktiveret  \\\hline
		\textbf{Hovedforløb}					
			&\begin{enumerate}
	
				\item \label{uc2login} Bruger logger ind med kode.
					%\textbf{[Undtagelse \ref{uc2login}a]} Bruger ér logget ind					
					
				\item Bruger vælger ''Aktiver'' i hovedmenu
										
				\item \label{uc2menu}UI viser mulige enheder samt ''Vælg alle'', ''Aktiver'' og ''Tilbage''
												
				\item Bruger markerer ønskede enheder til aktivering
												
				\item \label{uc2act} Bruger vælger ''Aktiver''\newline
					\textbf{[Undtagelse \ref{uc2act}a]} Bruger vælger ''Tilbage''
												
				\item \label{uc2sysact} Systemet aktiverer valgte enheder \newline
					\textbf{[Undtagelse \ref{uc2sysact}a]} Ingen valgte enheder
				
				\item UI viser besked om at enheder, er aktiverede
																	
				\item UI returnerer til hovedmenu
												
			\end{enumerate}\\ \hline
		
		\textbf{Undtagelser}	
		
		&\begin{enumerate}[label= \ref{uc2login}a.]
			\item Bruger skal ikke logge ind
		\end{enumerate}
						
		\begin{enumerate}[label= \ref{uc2act}a.]
			\item UI returnerer til hovedmenu og UC2 afbrydes
		\end{enumerate}						
							
		\begin{enumerate}[label= \ref{uc2sysact}a.]
			\item Hvis ingen unit er valgt udskrives en fejl på skærmen og beder brugeren om at vælge en unit og går til UC2.\ref{uc2menu}
		\end{enumerate} \\\hline
											
		%\textbf{Version}		&1.2 \\\hline

	\end{tabular}
	\label{UC2} 
\end{table}

% UC3: Deaktiver

\subsection{UC3: Deaktiver}
\begin{table}[H] \centering
\begin{tabular}{|p{6cm}|p{8cm}|}
	\hline
\textbf{Mål}	&
At brugeren kan deaktivere enkelte eller alle enheder, i systemet.
\\\hline
\textbf{Initialisering} &
Bruger vælger "deaktiver"
\\\hline
\textbf{Aktører og Stakeholders}	&
Bruger(Primær), Eksterne enheder(Sekundær)
\\\hline
\textbf{Referencer} &
UC1: Login
\\\hline
\textbf{Antal af samtidige hændelser} &
1
\\\hline
\textbf{Forudsætning} &
At systemet er helt eller delvist aktiveret.
\\\hline
\textbf{Efterfølgende tilstand} &
Enkelte eller alle enheder er deaktiveret
\\\hline
\textbf{Hovedforløb}	&
Bruger vælger deaktiver og følger instruktionerne på skærmen.

\begin{enumerate}
	
	\item Deaktiver alt

	\item Deaktiver alle låse
	
	\item Deaktiver babyalarm
	
\end{enumerate}

\\\hline
\textbf{Undtagelser}	&
Ingen
\\\hline
\textbf{Version}		&1.1 \\\hline
	\end{tabular}
	\label{tab:UC2} 
\end{table}

% UC4: Udlæs status

\subsection{UC4: Udlæs status}
\begin{table}[H] \centering
\begin{tabular}{|p{6cm}|p{8cm}|}
	\hline
\multicolumn{2}{|l|}{\textbf{UC4: Detekter brand}} \\\hline
\textbf{Mål}								&At detektere en opstået brand og eller røgudvikling \\\hline
\textbf{Initialisering}					& $CO_2$ \\\hline
\textbf{Aktører og Stakeholders}			&Primær: Bruger ønsker at få besked om brand \\\hline
\textbf{Referencer}						& Ingen \\\hline
\textbf{Antal af samtidige hændelser}	& 1 pr. sensor \\\hline
\textbf{Forudsætning}					& CSS sensor aktiv  \\\hline
\textbf{Efterfølgende tilstand}			& Besked til bruger - CSS sensor aktiv \\\hline
\textbf{Hovedforløb}						&  
1.CSS sensor aktiv \newline 
2. CSS sensor detekterer $CO_2$ \newline 
3. CSS sensor udløser alarm (alarm tilstand) \newline 
4. Bruger tvinger CSS sensor ud af alarm tilstand\\\hline
\textbf{Tilføjelser}						& Det er muligt at teste sensoren ved at trykke på en knap og herved "illustrere" en brand  \\\hline
%\textbf{Datavariationsliste}			&Test \\\hline
	\end{tabular}
	\label{UC4} 
\end{table}

% UC5: Detekter lyd

\subsection{UC5: Detekter lyd}
\begin{table}[H] \centering
\begin{tabular}{|p{6cm}|p{8cm}|}
	\hline
\multicolumn{2}{|l|}{\textbf{UC5: Detekter barn}} \\\hline
\textbf{Mål}								&At detektere om barnet bevæger sig eller græder \\\hline
\textbf{Initialisering}					&Barnet bevæger sig eller græder\\\hline
\textbf{Aktører og Stakeholders}			&Bruger(Primær): Ønsker at kunne overvåge barnet. SMS Bruger(Sekundær): 																	Modtager SMS ved gråd eller bevægelser. Barn(Sekundær): Ønskes overvåget 				 \\\hline
\textbf{Referencer}						&Advisering \\\hline
\textbf{Antal af samtidige hændelser}	&1 \\\hline
\textbf{Forudsætning}					&At CSS er aktiveret \\\hline
\textbf{Efterfølgende tilstand}			&Sensor stadig aktiv \\\hline
\textbf{Hovedforløb}						&\begin{enumerate}
	
				\item Systemet er aktiveret
												
				\item Systemet opfanger bevægelse eller gråd
												
				\item Systemet kalder advisering
								
			\end{enumerate}\\\hline1.
\textbf{Tilføjelser}					&Ingen \\\hline
%\textbf{Datavariationsliste}			&Test \\\hline
	\end{tabular}
	\label{UC5} 
\end{table}

% UC6: Rediger SMS-modtager

\subsection{UC6: Rediger SMS-modtager}
\begin{table}[H] \centering
\begin{tabular}{|p{6cm}|p{8cm}|}
	\hline
\multicolumn{2}{|l|}{\textbf{UC6: Detekter klima}} \\\hline
\textbf{Mål} &
Et system overvåger klimaet i et givet rum, og giver information hvis grænseværdier bliver overskredet  \\\hline

\textbf{Initialisering} &
 En grænseværdi overskrides og detekteres af systemet \\\hline
 
\textbf{Aktører og Stakeholders} &
Bruger(primær). SMS-bruger(sekundær) \\\hline

\textbf{Referencer} &
N/A \\\hline

\textbf{Antal af samtidige hændelser} &
Systemet kan overvåge på flere rum samtidig \\\hline

\textbf{Forudsætning} &
Systemet skal være tændt og aktivt. \\\hline

\textbf{Efterfølgende tilstand} &
Information er korret leveret videre til bruger og evt. SMS-bruger. \\\hline

\textbf{Hovedforløb} &
1. Systmet er aktivt

2. Overskridelse af grænseværdi

3. Overvågningen detekter overskridelsen 

4. Information sendes til bruger og evt. SMS-bruger
  \\\hline

\textbf{Tilføjelser} &
N/A \\\hline
%\textbf{Datavariationsliste}			&Test \\\hline
	\end{tabular}
	\label{UC6} 
\end{table}

% UC7: Startopsætning

\subsection{UC7: Startopsætning}
\begin{table}[H] \centering
\begin{tabular}{|p{6cm}|p{8cm}|}
	\hline
\multicolumn{2}{|l|}{\textbf{UC1: Ændre SMS bruger}} \\\hline
\textbf{Mål}								&
At brugeren(primær) kan ændre SMS-brugerens(sekundær) adviserings oplysninger i systemet.
 \\\hline
\textbf{Initialisering}					&
Initialisering sker via PC interface. Når brugeren(primær) vælger "Ændre SMS bruger" i menuen, bliver denne præsenteret for hvilket oplysninger der skal ændres ved SMS-brugeren(sekundær).
 \\\hline
\textbf{Aktører og Stakeholders}			&
Brugeren go SMS-brugeren
 \\\hline
\textbf{Referencer}						&
Ingen
 \\\hline
\textbf{Antal af samtidige hændelser}	&
Der kan fortages en hændelse af gangen.
 \\\hline
\textbf{Forudsætning}					&
At PC interfacet er aktivet og der er indtastet en adgangskode.
 \\\hline
\textbf{Efterfølgende tilstand}			&
Efter ændring af SMS-bruger, sendes brugeren tilbage til menuen i PC interfacet.
 \\\hline
\textbf{Hovedforløb}						&
\begin{enumerate}
\item Brugeren logger ind i systemet via PC interfacet.
\item Brugeren vælger "Ændre SMS-bruger" i menuen
\item Brugeren indtaster ændringer til SMS-brugeren og bekræfter.
\item Brugeren sendes tilbage til menuen.
\end{enumerate}
 \\\hline
\textbf{Tilføjelser}						&
Det vil kun være muligt at anvende danske mobilnummere \\\hline
%\textbf{Datavariationsliste}			&Test \\\hline
	\end{tabular}
	\label{UC7} 
\end{table}

% UC8: Tilføj/fjern X10 udtag

\subsection{UC8: Tilføj/fjern X10 udtag}
\begin{table}[H] \centering
\begin{tabular}{|p{6cm}|p{8cm}|}
	\hline
\multicolumn{2}{|l|}{\textbf{UC8: Advisering}} \\\hline
\textbf{Mål}								
&At brugeren kan opsætte/ændre systemets adviserings indstillinger
 \\\hline
\textbf{Initialisering}					
&Bruger vælger Advisering i interface
 \\\hline
\textbf{Aktører og Stakeholders}			
&Primær: Bruger
 \\\hline
\textbf{Referencer}						
&Login
 \\\hline
\textbf{Antal af samtidige hændelser}	
&Der kan fortages en ændring ad gangen
 \\\hline
\textbf{Forudsætning}					
&At interface er online
 \\\hline
\textbf{Efterfølgende tilstand}			
&Hovedmenu vises på skærmen.
 \\\hline
\textbf{Hovedforløb}						
& 
\begin{enumerate}

\item Bruger vælger advisering i interface

\item Brugeren har mulighed for at ændre adviserings indstillinger

\item Brugeren godkender
 
\end{enumerate}
 \\\hline

%\textbf{Datavariationsliste}			&Test \\\hline
	\end{tabular}
	\label{UC8} 
\end{table}

%% Ikke-funktionelle krav

\section{Ikke-funktionelle krav}
\subsection*{Usability}
\begin{itemize}
\item UI let at bruge
\subitem Skal kunne forstås efter gennemlæst manual.
\end{itemize}

\subsection*{Reliability}
\begin{itemize}
\item Levetid: 5 år uden hardware nedbrud
\item Software oppetid: Min. 1 måned før genstart
\end{itemize}

\subsection*{Performance}
\begin{itemize}
\item System respons må maksimalt være 2 +/- 0,5 sekunder
\item Startuptid fra power-off til funktionel tilstand maksimalt 2 +/- 0,5 minutter
\item Systemkapaciteten på max 15 CSS enheder
\item Ved alarm må der max gå 10 sek. før advisering
\item Ved alarm må der max gå 1 min før SMS advisering
\end{itemize}

\subsection*{Supportability}
\begin{itemize}
\item CSS enheder kan udskiftes separat ved simpel omkodning vha. dipswitches 
\item Systemet er plug’n’play i en almindelig husholdning
\item CSS enheder kan tilføjes og installeres løbende
\end{itemize}

\subsection*{Generelle krav}
\begin{itemize}
\item Systemet skal virke på det eksisterende 230 Vac netværk i almindelige husstande
\item Systemet skal kommunikere på X10 protokollen
\item Systemet skal kunne afsende SMS adviseringer
\end{itemize}

\subsection*{CSS enheder}
\begin{itemize}
\item Outlet enheder skal kunne være i en 1,5 moduls Fuga stikdåse
\item Enheder skal have en LED indikator som viser at den er aktiv
\item CSS enhederne skal køre på 230 Vac/13 A
\end{itemize}

\subsection*{Eksterne enheder}
\begin{itemize}
\item Lyddetektoren skal registrere lyde på over 68 dB.
\item Låse enheder må maks. være 8x5x3 cm
\item Låse enhederne skal kunne holde X antal kilo.
\end{itemize}


%% Begrænsninger

\section{Begrænsninger} \label{sectionBegraensninger}
\begin{itemize}
	\item Prototypen udføres i et 18 Vac testmiljø
	\item I stedet for magnetlåse til at simulere låsemekanismen bruges en lysindikator
	\item Protypen udføres med et STK500 kit, hvorfor krav til dimensionerne frafalder
\end{itemize}

\newpage
\section{HMI(Human Machine Interface)}

Billederne er inverteret for læsbarhedens skyld.

\begin{figure}[h] \centering
{\includegraphics[width=0.75\textwidth]{billeder/cmdprompt/CSS_menu}}
\caption{CSS Menu}
%\label{lab:CSS Menu}
\end{figure}

\begin{figure}[h] \centering
{\includegraphics[width=0.75\textwidth]{billeder/cmdprompt/CSS_pre_login_menu}}
\caption{CSS Pre-Login}
%\label{fig:CSS Login}
\end{figure}

\begin{figure}[h] \centering
{\includegraphics[width=0.75\textwidth]{billeder/cmdprompt/CSS_login}}
\caption{CSS Login}
%\label{fig:CSS Login}
\end{figure}

\begin{figure}[h] \centering
{\includegraphics[width=0.75\textwidth]{billeder/cmdprompt/CSS_aktiver}}
\caption{CSS Aktiver}
%\label{fig:CSS Aktiver}
\end{figure}

\begin{figure}[h] \centering
{\includegraphics[width=0.75\textwidth]{billeder/cmdprompt/CSS_deaktiver}}
\caption{CSS Deaktvier}
%\label{fig:CSS Deaktvier}
\end{figure}

\begin{figure}[h] \centering
{\includegraphics[width=0.75\textwidth]{billeder/cmdprompt/CSS_vis_status}}
\caption{CSS Vis Status}
%\label{fig:CSS Vis Status}
\end{figure}

\begin{figure}[h] \centering
{\includegraphics[width=0.75\textwidth]{billeder/cmdprompt/CSS_advisering}}
\caption{CSS Advisering}
%\label{fig:CSS Advisering}
\end{figure}

\begin{figure}[h] \centering
{\includegraphics[width=0.75\textwidth]{billeder/cmdprompt/CSS_startopsaetning}}
\caption{CSS Startopsætning}
%\label{fig:CSS Advisering}
\end{figure}

