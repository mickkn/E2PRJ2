\chapter{Kravspecifikation}

\section{Aktører}

\begin{figure}[h] \centering
\fbox{\includegraphics[width=0.6\textwidth]{billeder/Kontekst_Diagram}}
\caption{Kontekst diagram}
\label{lab:kontekstdiagram}
\end{figure}

\subsection{Bruger}
\begin{table}[htbp] \centering
\begin{tabular}{|p{4cm}|p{7cm}|}
	\hline
\textbf{Aktørnavn} &Bruger \\\hline
\textbf{Type Beskrivelse} &
Bruger aktøren er ejeren af systemet eller den voksne med adgang til Computeren.
Dette kunne være, forældre, barnepige osv.	
\\\hline
	\end{tabular}
\end{table}

\subsection{Barn}
\begin{table}[htbp] \centering
\begin{tabular}{|p{4cm}|p{7cm}|}
	\hline
\textbf{Aktørnavn} &Barn \\\hline
\textbf{Type Beskrivelse} &
Barnet eller børnene i huset, som systemet skal beskytte.	
\\\hline
	\end{tabular}
\end{table}

\subsection{SMS Bruger}
\begin{table}[htbp] \centering
\begin{tabular}{|p{4cm}|p{7cm}|}
	\hline
\textbf{Aktørnavn} &SMS Bruger \\\hline
\textbf{Type Beskrivelse} &
Ligesom Bruger (ejeren, forældrene osv.)
Men kan også være naboen eller et familiemedlem der bor i nærheden.
\\\hline
	\end{tabular}
\end{table}

\clearpage		% For at fikse Diagram inde i Section

\section{Usecases}

\begin{figure}[htbp] \centering
\vspace*{\fill}
\includegraphics[width=\textwidth]{billeder/Usecase_Diagram}
\caption{Usecase diagram}
\label{lab:usecasediagram}
\vspace*{\fill}
\end{figure}

\subsection{Usecase 1}
\begin{table}[H] \centering
\begin{tabular}{|p{6cm}|p{8cm}|}
	\hline
<<<<<<< HEAD
\textbf{Mål}								
&At tilmeldt bruger af systemet kan logge ind ved brug af personlig brugernavn og password
 \\\hline
\textbf{Initialisering}					
&Bruger vælger login i interface
 \\\hline
\textbf{Aktører og Stakeholders}			
&Primær: Bruger
 \\\hline
\textbf{Referencer}						
&Ingen
 \\\hline
\textbf{Antal af samtidige hændelser}	
&Der kan fortages ét login ad gangen (sådan skal det formuleres!)
 \\\hline
\textbf{Forudsætning}					
&At interface er online
 \\\hline
\textbf{Efterfølgende tilstand}			
&At bruger er logget ind og hovedmenu vises på skærmen. Hele systmet er klar til brug
 \\\hline
\textbf{Hovedforløb}						
& 
\begin{enumerate}

\item Bruger vælger login i interface

\item \label{UC8und1}Bruger indtaster personlig brugernavn og adgangskode [Undtagelse 1: Bruger vælger Annuller]

\item \label{UC8und2} Systemet validerer brugernavn og adgangskode [Undtagelse 2: Ikke valideret]

\item Bruger får adgang til hovedmenu
 
\end{enumerate}
\\\hline

\textbf{Undtagelser}						
&\begin{enumerate}[label= \ref{UC8und1}a.]
			\item Bruger vælger annuller og kommer tilbage til startskærm
		\end{enumerate}
											
		\begin{enumerate}[label= \ref{UC8und2}a.]
			\item Brugernavn eller adgangskode ikke indtastet korret. Brugernavn og adganskode indtastes igen.
=======
		\multicolumn{2}{|l|}{\textbf{UC1: Aktiver CSS enhed(er)}} \\\hline
		
		\textbf{Mål}							&At brugeren kan aktivere enkelte eller alle enheder, i systemet.	\\\hline
		\textbf{Initialisering}				&Bruger vælger "Aktiver". 										\\\hline
		\textbf{Aktører og Stakeholders}		&Primær: Bruger ønsker at aktivere CSS enheder					\\\hline
		\textbf{Referencer}					&Login															\\\hline
		\textbf{Antal af samtidige hændelser}&1 																\\\hline
		\textbf{Forudsætning}				&Ingen															\\\hline
		\textbf{Efterfølgende tilstand}		&Hovedmenu vises 												\\\hline
		\textbf{Hovedforløb}					
			&\begin{enumerate}
	
				\item Bruger trykker på "Aktiver" knap
				
				\item Bruger logger ind med kode.
										
				\item Interface viser mulige enheder samt "Vælg alle", "Aktiver" og "Tilbage"-knapper
												
				\item \label{uc1select} Bruger markerer ønskede enheder til aktivering
												
				\item \label{uc1ex1} Bruger trykker "Aktiver"\newline
					\textbf{[Undtagelse \ref{uc1ex1}a]} Bruger trykker "Tilbage"
												
				\item \label{uc1ex2} Systemet aktiverer valgte enheder\newline
					\textbf{[Undtagelse \ref{uc1ex2}a]} Ingen valgte enheder
				
				\item Brugerinterface viser besked om at enheder, er aktiverede
																	
				\item Interface returnerer til hovedmenu
												
			\end{enumerate}\\\hline
		
		\textbf{Undtagelser}					
		&\begin{enumerate}[label= \ref{uc1ex1}a.]
			\item Brugerinterface returnerer til standardskærm og UC1 afbrydes
		\end{enumerate}
											
		\begin{enumerate}[label= \ref{uc1ex2}a.]
			\item Hvis ingen unit er valgt udskrives en fejl på skærmen og beder brugeren om at vælge en enhed og går til UC1.\ref{uc1select}.
>>>>>>> FETCH_HEAD
		\end{enumerate} \\\hline


		\textbf{Version}		&1.0 \\\hline
	\end{tabular}
	\label{UC1} 
\end{table}

\subsection{Usecase 2}
\begin{table}[H] \centering
	\begin{tabular} {|p{6cm}|p{8cm}|}
	\hline		
		\textbf{Mål}							&At Bruger kan aktivere enkelte eller alle enheder, i systemet\\\hline
		\textbf{Initialisering}				&Bruger vælger "Aktiver" 	\\\hline
		\textbf{Aktører og Stakeholders}		&Bruger(Primær) 				\\\hline
		\textbf{Referencer}					&UC1: Login					\\\hline
		\textbf{Antal af samtidige hændelser}&1 							\\\hline
		\textbf{Forudsætning}				&Bruger er logget ind i systemet\\\hline
		\textbf{Efterfølgende tilstand}		&Enkelte eller alle enheder er aktiveret  \\\hline
		\textbf{Hovedforløb}					
			&\begin{enumerate}
	
				\item \label{uc2login} Bruger logger ind med kode.
					%\textbf{[Undtagelse \ref{uc2login}a]} Bruger ér logget ind					
					
				\item Bruger vælger ''Aktiver'' i hovedmenu
										
				\item \label{uc2menu}UI viser mulige enheder samt ''Vælg alle'', ''Aktiver'' og ''Tilbage''
												
				\item Bruger markerer ønskede enheder til aktivering
												
				\item \label{uc2act} Bruger vælger ''Aktiver''\newline
					\textbf{[Undtagelse \ref{uc2act}a]} Bruger vælger ''Tilbage''
												
				\item \label{uc2sysact} Systemet aktiverer valgte enheder \newline
					\textbf{[Undtagelse \ref{uc2sysact}a]} Ingen valgte enheder
				
				\item UI viser besked om at enheder, er aktiverede
																	
				\item UI returnerer til hovedmenu
												
			\end{enumerate}\\ \hline
		
		\textbf{Undtagelser}	
		
		&\begin{enumerate}[label= \ref{uc2login}a.]
			\item Bruger skal ikke logge ind
		\end{enumerate}
						
		\begin{enumerate}[label= \ref{uc2act}a.]
			\item UI returnerer til hovedmenu og UC2 afbrydes
		\end{enumerate}						
							
		\begin{enumerate}[label= \ref{uc2sysact}a.]
			\item Hvis ingen unit er valgt udskrives en fejl på skærmen og beder brugeren om at vælge en unit og går til UC2.\ref{uc2menu}
		\end{enumerate} \\\hline
											
		%\textbf{Version}		&1.2 \\\hline

	\end{tabular}
	\label{UC2} 
\end{table}

\subsection{Usecase 3}
\begin{table}[H] \centering
\begin{tabular}{|p{6cm}|p{8cm}|}
	\hline
\textbf{Mål}	&
At brugeren kan deaktivere enkelte eller alle enheder, i systemet.
\\\hline
\textbf{Initialisering} &
Bruger vælger "deaktiver"
\\\hline
\textbf{Aktører og Stakeholders}	&
Bruger(Primær), Eksterne enheder(Sekundær)
\\\hline
\textbf{Referencer} &
UC1: Login
\\\hline
\textbf{Antal af samtidige hændelser} &
1
\\\hline
\textbf{Forudsætning} &
At systemet er helt eller delvist aktiveret.
\\\hline
\textbf{Efterfølgende tilstand} &
Enkelte eller alle enheder er deaktiveret
\\\hline
\textbf{Hovedforløb}	&
Bruger vælger deaktiver og følger instruktionerne på skærmen.

\begin{enumerate}
	
	\item Deaktiver alt

	\item Deaktiver alle låse
	
	\item Deaktiver babyalarm
	
\end{enumerate}

\\\hline
\textbf{Undtagelser}	&
Ingen
\\\hline
\textbf{Version}		&1.1 \\\hline
	\end{tabular}
	\label{tab:UC2} 
\end{table}

\subsection{Usecase 4}
\begin{table}[H] \centering
\begin{tabular}{|p{6cm}|p{8cm}|}
	\hline
\multicolumn{2}{|l|}{\textbf{UC4: Detekter brand}} \\\hline
\textbf{Mål}								&At detektere en opstået brand og eller røgudvikling \\\hline
\textbf{Initialisering}					& $CO_2$ \\\hline
\textbf{Aktører og Stakeholders}			&Primær: Bruger ønsker at få besked om brand \\\hline
\textbf{Referencer}						& Ingen \\\hline
\textbf{Antal af samtidige hændelser}	& 1 pr. sensor \\\hline
\textbf{Forudsætning}					& CSS sensor aktiv  \\\hline
\textbf{Efterfølgende tilstand}			& Besked til bruger - CSS sensor aktiv \\\hline
\textbf{Hovedforløb}						&  
1.CSS sensor aktiv \newline 
2. CSS sensor detekterer $CO_2$ \newline 
3. CSS sensor udløser alarm (alarm tilstand) \newline 
4. Bruger tvinger CSS sensor ud af alarm tilstand\\\hline
\textbf{Tilføjelser}						& Det er muligt at teste sensoren ved at trykke på en knap og herved "illustrere" en brand  \\\hline
%\textbf{Datavariationsliste}			&Test \\\hline
	\end{tabular}
	\label{UC4} 
\end{table}

\subsection{Usecase 5}
\begin{table}[H] \centering
\begin{tabular}{|p{6cm}|p{8cm}|}
	\hline
\multicolumn{2}{|l|}{\textbf{UC5: Detekter barn}} \\\hline
\textbf{Mål}								&At detektere om barnet bevæger sig eller græder \\\hline
\textbf{Initialisering}					&Barnet bevæger sig eller græder\\\hline
\textbf{Aktører og Stakeholders}			&Bruger(Primær): Ønsker at kunne overvåge barnet. SMS Bruger(Sekundær): 																	Modtager SMS ved gråd eller bevægelser. Barn(Sekundær): Ønskes overvåget 				 \\\hline
\textbf{Referencer}						&Advisering \\\hline
\textbf{Antal af samtidige hændelser}	&1 \\\hline
\textbf{Forudsætning}					&At CSS er aktiveret \\\hline
\textbf{Efterfølgende tilstand}			&Sensor stadig aktiv \\\hline
\textbf{Hovedforløb}						&\begin{enumerate}
	
				\item Systemet er aktiveret
												
				\item Systemet opfanger bevægelse eller gråd
												
				\item Systemet kalder advisering
								
			\end{enumerate}\\\hline1.
\textbf{Tilføjelser}					&Ingen \\\hline
%\textbf{Datavariationsliste}			&Test \\\hline
	\end{tabular}
	\label{UC5} 
\end{table}

\subsection{Usecase 6}
\begin{table}[H] \centering
\begin{tabular}{|p{6cm}|p{8cm}|}
	\hline
\multicolumn{2}{|l|}{\textbf{UC6: Detekter klima}} \\\hline
\textbf{Mål} &
Et system overvåger klimaet i et givet rum, og giver information hvis grænseværdier bliver overskredet  \\\hline

\textbf{Initialisering} &
 En grænseværdi overskrides og detekteres af systemet \\\hline
 
\textbf{Aktører og Stakeholders} &
Bruger(primær). SMS-bruger(sekundær) \\\hline

\textbf{Referencer} &
N/A \\\hline

\textbf{Antal af samtidige hændelser} &
Systemet kan overvåge på flere rum samtidig \\\hline

\textbf{Forudsætning} &
Systemet skal være tændt og aktivt. \\\hline

\textbf{Efterfølgende tilstand} &
Information er korret leveret videre til bruger og evt. SMS-bruger. \\\hline

\textbf{Hovedforløb} &
1. Systmet er aktivt

2. Overskridelse af grænseværdi

3. Overvågningen detekter overskridelsen 

4. Information sendes til bruger og evt. SMS-bruger
  \\\hline

\textbf{Tilføjelser} &
N/A \\\hline
%\textbf{Datavariationsliste}			&Test \\\hline
	\end{tabular}
	\label{UC6} 
\end{table}

\subsection{Usecase 7}
\begin{table}[H] \centering
\begin{tabular}{|p{6cm}|p{8cm}|}
	\hline
\multicolumn{2}{|l|}{\textbf{UC1: Ændre SMS bruger}} \\\hline
\textbf{Mål}								&
At brugeren(primær) kan ændre SMS-brugerens(sekundær) adviserings oplysninger i systemet.
 \\\hline
\textbf{Initialisering}					&
Initialisering sker via PC interface. Når brugeren(primær) vælger "Ændre SMS bruger" i menuen, bliver denne præsenteret for hvilket oplysninger der skal ændres ved SMS-brugeren(sekundær).
 \\\hline
\textbf{Aktører og Stakeholders}			&
Brugeren go SMS-brugeren
 \\\hline
\textbf{Referencer}						&
Ingen
 \\\hline
\textbf{Antal af samtidige hændelser}	&
Der kan fortages en hændelse af gangen.
 \\\hline
\textbf{Forudsætning}					&
At PC interfacet er aktivet og der er indtastet en adgangskode.
 \\\hline
\textbf{Efterfølgende tilstand}			&
Efter ændring af SMS-bruger, sendes brugeren tilbage til menuen i PC interfacet.
 \\\hline
\textbf{Hovedforløb}						&
\begin{enumerate}
\item Brugeren logger ind i systemet via PC interfacet.
\item Brugeren vælger "Ændre SMS-bruger" i menuen
\item Brugeren indtaster ændringer til SMS-brugeren og bekræfter.
\item Brugeren sendes tilbage til menuen.
\end{enumerate}
 \\\hline
\textbf{Tilføjelser}						&
Det vil kun være muligt at anvende danske mobilnummere \\\hline
%\textbf{Datavariationsliste}			&Test \\\hline
	\end{tabular}
	\label{UC7} 
\end{table}

% Advisering

\subsection{Usecase 8}
\begin{table}[H] \centering
\begin{tabular}{|p{6cm}|p{8cm}|}
	\hline
\multicolumn{2}{|l|}{\textbf{UC8: Advisering}} \\\hline
\textbf{Mål}								
&At brugeren kan opsætte/ændre systemets adviserings indstillinger
 \\\hline
\textbf{Initialisering}					
&Bruger vælger Advisering i interface
 \\\hline
\textbf{Aktører og Stakeholders}			
&Primær: Bruger
 \\\hline
\textbf{Referencer}						
&Login
 \\\hline
\textbf{Antal af samtidige hændelser}	
&Der kan fortages en ændring ad gangen
 \\\hline
\textbf{Forudsætning}					
&At interface er online
 \\\hline
\textbf{Efterfølgende tilstand}			
&Hovedmenu vises på skærmen.
 \\\hline
\textbf{Hovedforløb}						
& 
\begin{enumerate}

\item Bruger vælger advisering i interface

\item Brugeren har mulighed for at ændre adviserings indstillinger

\item Brugeren godkender
 
\end{enumerate}
 \\\hline

%\textbf{Datavariationsliste}			&Test \\\hline
	\end{tabular}
	\label{UC8} 
\end{table}

\subsection{Usecase 9}
\input{filer/kravspec/uc9}
