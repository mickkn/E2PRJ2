\begin{center} \centering
	\begin{longtable}{|p{6cm}|p{8cm}|}
	\hline
		\multicolumn{2}{|l|}{\textbf{UC1: Login}} \\\hline
		\endfirsthead
		
		\multicolumn{2}{l}{...fortsat fra forrige side} \\ \hline 
		\multicolumn{2}{|l|}{\textbf{UC1: Login}} \\\hline
		\endhead		

\textbf{Mål}								
&At Bruger kan logge ind ved hjælp af adgangskode
 \\\hline
\textbf{Initialisering}					
&Bruger vælger login i interface
 \\\hline
\textbf{Aktører og Stakeholders}			
&Bruger(Primær), DE2-board(Sekundær)
 \\\hline
\textbf{Referencer}						
&Ingen
 \\\hline
\textbf{Antal af samtidige hændelser}	
&1
 \\\hline
\textbf{Forudsætning}					
&At interfacet er tændt
 \\\hline
\textbf{Efterfølgende tilstand}			
&At bruger er logget ind og hovedmenu vises på skærmen. Hele systemet er klar til brug
 \\\hline
\textbf{Hovedforløb}						
& 
\begin{enumerate}

\item Bruger vælger login i interfacet

\item \label{UC2und1}Bruger indtaster 3 koder adskilt af ''Enter'' på DE2-board \newline
\textbf{[Undtagelse \ref{UC2und1}a]} Bruger vælger annuller

\item Bruger får adgang til hovedmenuen	
 
\end{enumerate}
\\\hline

\textbf{Undtagelser}						
&\begin{enumerate}[label= \ref{UC2und1}a.]
			\item Bruger vælger annuller og kommer tilbage til loginskærm
		\end{enumerate}
\\\hline


		%\textbf{Version}		&1.0 \\\hline
	\end{longtable}
	\label{UC1} 
\end{center}