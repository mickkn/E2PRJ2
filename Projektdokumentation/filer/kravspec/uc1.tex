\begin{table}[H] \centering
\begin{tabular}{|p{6cm}|p{8cm}|}
	\hline
<<<<<<< HEAD
\textbf{Mål}								
&At tilmeldt bruger af systemet kan logge ind ved brug af personlig brugernavn og password
 \\\hline
\textbf{Initialisering}					
&Bruger vælger login i interface
 \\\hline
\textbf{Aktører og Stakeholders}			
&Primær: Bruger
 \\\hline
\textbf{Referencer}						
&Ingen
 \\\hline
\textbf{Antal af samtidige hændelser}	
&Der kan fortages ét login ad gangen (sådan skal det formuleres!)
 \\\hline
\textbf{Forudsætning}					
&At interface er online
 \\\hline
\textbf{Efterfølgende tilstand}			
&At bruger er logget ind og hovedmenu vises på skærmen. Hele systmet er klar til brug
 \\\hline
\textbf{Hovedforløb}						
& 
\begin{enumerate}

\item Bruger vælger login i interface

\item \label{UC8und1}Bruger indtaster personlig brugernavn og adgangskode [Undtagelse 1: Bruger vælger Annuller]

\item \label{UC8und2} Systemet validerer brugernavn og adgangskode [Undtagelse 2: Ikke valideret]

\item Bruger får adgang til hovedmenu
 
\end{enumerate}
\\\hline

\textbf{Undtagelser}						
&\begin{enumerate}[label= \ref{UC8und1}a.]
			\item Bruger vælger annuller og kommer tilbage til startskærm
		\end{enumerate}
											
		\begin{enumerate}[label= \ref{UC8und2}a.]
			\item Brugernavn eller adgangskode ikke indtastet korret. Brugernavn og adganskode indtastes igen.
=======
		\multicolumn{2}{|l|}{\textbf{UC1: Aktiver CSS enhed(er)}} \\\hline
		
		\textbf{Mål}							&At brugeren kan aktivere enkelte eller alle enheder, i systemet.	\\\hline
		\textbf{Initialisering}				&Bruger vælger "Aktiver". 										\\\hline
		\textbf{Aktører og Stakeholders}		&Primær: Bruger ønsker at aktivere CSS enheder					\\\hline
		\textbf{Referencer}					&Login															\\\hline
		\textbf{Antal af samtidige hændelser}&1 																\\\hline
		\textbf{Forudsætning}				&Ingen															\\\hline
		\textbf{Efterfølgende tilstand}		&Hovedmenu vises 												\\\hline
		\textbf{Hovedforløb}					
			&\begin{enumerate}
	
				\item Bruger trykker på "Aktiver" knap
				
				\item Bruger logger ind med kode.
										
				\item Interface viser mulige enheder samt "Vælg alle", "Aktiver" og "Tilbage"-knapper
												
				\item \label{uc1select} Bruger markerer ønskede enheder til aktivering
												
				\item \label{uc1ex1} Bruger trykker "Aktiver"\newline
					\textbf{[Undtagelse \ref{uc1ex1}a]} Bruger trykker "Tilbage"
												
				\item \label{uc1ex2} Systemet aktiverer valgte enheder\newline
					\textbf{[Undtagelse \ref{uc1ex2}a]} Ingen valgte enheder
				
				\item Brugerinterface viser besked om at enheder, er aktiverede
																	
				\item Interface returnerer til hovedmenu
												
			\end{enumerate}\\\hline
		
		\textbf{Undtagelser}					
		&\begin{enumerate}[label= \ref{uc1ex1}a.]
			\item Brugerinterface returnerer til standardskærm og UC1 afbrydes
		\end{enumerate}
											
		\begin{enumerate}[label= \ref{uc1ex2}a.]
			\item Hvis ingen unit er valgt udskrives en fejl på skærmen og beder brugeren om at vælge en enhed og går til UC1.\ref{uc1select}.
>>>>>>> FETCH_HEAD
		\end{enumerate} \\\hline


		\textbf{Version}		&1.0 \\\hline
	\end{tabular}
	\label{UC1} 
\end{table}