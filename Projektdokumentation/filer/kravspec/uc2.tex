\begin{center} \centering
	\begin{longtable}{|p{6cm}|p{8cm}|}
	\hline
		\multicolumn{2}{|l|}{\textbf{UC2: Aktiver}} \\\hline
		\endfirsthead
		
		\multicolumn{2}{l}{...fortsat fra forrige side} \\ \hline 
		\multicolumn{2}{|l|}{\textbf{UC2: Aktiver}} \\\hline
		\endhead	
		
		\textbf{Mål}							&At Bruger kan aktivere enkelte eller alle enheder, i systemet\\\hline
		\textbf{Initialisering}				&Bruger vælger ''Aktiver'' hovedmenu  	\\\hline
		\textbf{Aktører og Stakeholders}		&Bruger(Primær), Eksterne enheder(Sekundær) 		\\\hline
		\textbf{Referencer}					&UC1: Login					\\\hline
		\textbf{Antal af samtidige hændelser}&1 							\\\hline
		\textbf{Forudsætning}				&Bruger er logget ind (UC1: Login)\\\hline
		\textbf{Efterfølgende tilstand}		&Enkelte eller alle enheder er aktiveret  \\\hline
		\textbf{Hovedforløb}					
			&\begin{enumerate}				
					
				\item Bruger vælger ''Aktiver'' i hovedmenu
										
				\item \label{uc2menu}UI viser mulige enheder samt ''Aktiver'' og ''Tilbage''
												
				\item \label{uc2act} Bruger vælger ''Aktiver''\newline
					\textbf{[Undtagelse \ref{uc2act}a]} Bruger vælger ''Tilbage''
												
				\item \label{uc2sysact} Systemet aktiverer valgte enheder
				
				\item UI viser besked om at enheder, er aktiverede
																	
				\item UI returnerer til hovedmenu
												
			\end{enumerate}\\ \hline
		
		\textbf{Undtagelser}	
		
		&\begin{enumerate}[label= \ref{uc2act}a.]
			\item UI returnerer til hovedmenu og UC2 afbrydes
		\end{enumerate} \\\hline
											
		%\textbf{Version}		&1.2 \\\hline

	\end{longtable}
	\label{UC2} 
\end{center}