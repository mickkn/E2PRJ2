\begin{table}[H] \centering
	\begin{tabular} {|p{6cm}|p{8cm}|}
	\hline		
		\textbf{Mål}							&At brugeren kan aktivere enkelte eller alle enheder, i systemet.	\\\hline
		\textbf{Initialisering}				&Bruger vælger "Aktiver" 								\\\hline
		\textbf{Aktører og Stakeholders}		&Bruger(Primær) 							\\\hline
		\textbf{Referencer}					&UC1: Login														\\\hline
		\textbf{Antal af samtidige hændelser}&1 																\\\hline
		\textbf{Forudsætning}				&Ingen														\\\hline
		\textbf{Efterfølgende tilstand}		&Enkelte eller alle enheder er aktiveret  								\\\hline
		\textbf{Hovedforløb}					
			&\begin{enumerate}
	
				\item Bruger logger ind med kode.
					
				\item Bruger vælger "Aktiver" i hovedmenu
										
				\item UI viser mulige enheder samt "Vælg alle", "Aktiver" og "Tilbage"
												
				\item \label{uc1select} Bruger markerer ønskede enheder til aktivering
												
				\item \label{uc1ex1} Bruger trykker "Aktiver"\newline
					\textbf{[Undtagelse \ref{uc1ex1}a]} Bruger trykker "Tilbage"
												
				\item \label{uc1ex2} Systemet aktiverer valgte enheder\newline
					\textbf{[Undtagelse \ref{uc1ex2}a]} Ingen valgte enheder
				
				\item UI viser besked om at enheder, er aktiverede
																	
				\item UI returnerer til hovedmenu
												
			\end{enumerate}\\\hline
		
		\textbf{Undtagelser}					
		&\begin{enumerate}[label= \ref{uc1ex1}a.]
			\item Brugerinterface returnerer til standardskærm og UC1 afbrydes
		\end{enumerate}
											
		\begin{enumerate}[label= \ref{uc1ex2}a.]
			\item Hvis ingen unit er valgt udskrives en fejl på skærmen og beder brugeren om at vælge en unit og går til UC1.\ref{uc1select}.
		\end{enumerate} \\\hline
											
		\textbf{Version}		&1.1 \\\hline

	\end{tabular}
	\label{UC2} 
\end{table}