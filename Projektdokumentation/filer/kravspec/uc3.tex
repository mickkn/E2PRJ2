\begin{center} \centering
	\begin{longtable}{|p{6cm}|p{8cm}|}
	\hline
		\multicolumn{2}{|l|}{\textbf{UC3: Deaktiver}} \\\hline
		\endfirsthead
		
		\multicolumn{2}{l}{...fortsat fra forrige side} \\ \hline 
		\multicolumn{2}{|l|}{\textbf{UC3: Deaktiver}} \\\hline
		\endhead	

		\textbf{Mål}						&At Bruger kan deaktivere enkelte eller alle enheder, i systemet. \\\hline
		\textbf{Initialisering} 			&Bruger vælger ''Deaktiver'' \\ \hline
		\textbf{Aktører og Stakeholders}	&Bruger(Primær), Eksterne enheder(Sekundær)\\ \hline
		\textbf{Referencer} 				&UC1: Login \\ \hline
		\textbf{Antal af samtidige hændelser} &1 \\ \hline
		\textbf{Forudsætning} 			&Bruger er logget ind (UC1: Login)\\ \hline
		\textbf{Efterfølgende tilstand} 	&Enkelte eller alle enheder er deaktiveret \\ \hline
		\textbf{Hovedforløb}				&

	\begin{enumerate}	
						
					
				\item Bruger vælger ''Deaktiver'' i hovedmenu
										
				\item \label{uc3menu}UI viser mulige enheder samt ''Deaktiver''  og ''Tilbage''
												
				\item \label{uc3deact} Bruger vælger ''Deaktiver''\newline
					\textbf{[Undtagelse \ref{uc3deact}a]} Bruger vælger ''Tilbage''
												
				\item \label{uc3sysdeact} Systemet deaktiverer valgte enheder
				
				\item UI viser besked om at enheder, er deaktiverede
																	
				\item UI returnerer til hovedmenu	
	
	\end{enumerate} \\ \hline

		\textbf{Undtagelser}	
		
		&\begin{enumerate}[label= \ref{uc3deact}a.]
			\item UI returnerer til hovedmenu og UC3 afbrydes
		\end{enumerate}	
		
		\\ \hline					
											
		%\textbf{Version}		&1.2 \\\hline
		
	\end{longtable} 
	\label{UC3}
\end{center}