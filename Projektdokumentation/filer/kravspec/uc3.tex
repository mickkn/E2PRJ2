\begin{table}[H] \centering
	\label{tab:UC3}
\begin{tabular}{|p{6cm}|p{8cm}|}
	\hline
		\textbf{Mål}						&At Bruger kan deaktivere enkelte eller alle enheder, i systemet. \\\hline
		\textbf{Initialisering} 			&Bruger vælger ''Deaktiver'' \\ \hline
		\textbf{Aktører og Stakeholders}	&Bruger(Primær), Eksterne enheder(Sekundær)\\ \hline
		\textbf{Referencer} 				&UC1: Login \\ \hline
		\textbf{Antal af samtidige hændelser} &1 \\ \hline
		\textbf{Forudsætning} 			&At systemet er helt eller delvist aktiveret.\\ \hline
		\textbf{Efterfølgende tilstand} 	&Enkelte eller alle enheder er deaktiveret \\ \hline
		\textbf{Hovedforløb}				&

	\begin{enumerate}	
	
				\item \label{uc3login} Bruger logger ind med kode. \newline
					\textbf{[Undtagelse \ref{uc3login}a]} Bruger ér logget ind					
					
				\item Bruger vælger "Deaktiver" i hovedmenu
										
				\item \label{uc3menu}UI viser mulige enheder samt ''Vælg alle'', ''Deaktiver''  og ''Tilbage''
												
				\item Bruger markerer ønskede enheder til deaktivering
												
				\item \label{uc3deact} Bruger vælger "Deaktiver"\newline
					\textbf{[Undtagelse \ref{uc3deact}a]} Bruger vælger ''Tilbage''
												
				\item \label{uc3sysdeact} Systemet deaktiverer valgte enheder \newline
					\textbf{[Undtagelse \ref{uc3sysdeact}a]} Ingen valgte enheder
				
				\item UI viser besked om at enheder, er deaktiverede
																	
				\item UI returnerer til hovedmenu	
	
	\end{enumerate} \\ \hline

		\textbf{Undtagelser}	
		
		&\begin{enumerate}[label= \ref{uc3login}a.]
			\item Hovedmenu vises
		\end{enumerate}
						
		\begin{enumerate}[label= \ref{uc3deact}a.]
			\item UI returnerer til hovedmenu og UC3 afbrydes
		\end{enumerate}						
							
		\begin{enumerate}[label= \ref{uc3sysdeact}a.]
			\item Hvis ingen enheder er valgt udskrives en fejl på skærmen og beder brugeren om at vælge en enhed og går til UC3.\ref{uc3menu}
		\end{enumerate} \\\hline
											
		%\textbf{Version}		&1.2 \\\hline
		
	\end{tabular} 
\end{table}