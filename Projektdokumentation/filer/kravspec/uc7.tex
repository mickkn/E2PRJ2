\begin{center} \centering
	\begin{longtable}{|p{6cm}|p{8cm}|}
	\hline
		\multicolumn{2}{|l|}{\textbf{UC7: Startopsætning}} \\\hline
		\endfirsthead
		
		\multicolumn{2}{l}{...fortsat fra forrige side} \\ \hline 
		\multicolumn{2}{|l|}{\textbf{UC7: Startopsætning}} \\\hline
		\endhead	
		
		\textbf{Mål}							&At brugeren kan opsætte systemet første gang.					\\\hline
		\textbf{Initialisering}				&Bruger	starter systemet første gang				 										\\\hline
		\textbf{Aktører og Stakeholders}		&Bruger(Primær)			\\\hline
		\textbf{Referencer}					&UC8: Tilføj/Fjern X10 udtag										\\\hline
		\textbf{Antal af samtidige hændelser}&1 																\\\hline
		\textbf{Forudsætning}				&Ingen															\\\hline
		\textbf{Efterfølgende tilstand}		&Systemet er fuldt opsat											\\\hline
		\textbf{Hovedforløb}					
			&\begin{enumerate}
	
				\item Bruger sætter følgende kabler sammen:
				
					\subitem Serielt RS-232 kabel mellem hovedenhedens COM-port og computer
					\subitem Medfølgende styrekabel til lyddetektor forbindes mellem hovedenhed og lyddetektor
					\subitem Strømkabel fra et ledigt 230 Vac udtag til hovedenhedens AC indgang
				
				\item Bruger tænder for hovedenhed og computer på Tænd/Sluk knappen
				
				\item CSS programmet startes på computeren (UC1: Login gennemføres)
				
				\item \label{uc7tilfoj} UC8: Tilføj/fjern X10 udtag udføres
				
				\item Punkt \ref{uc7tilfoj} gentages med antallet af X10 udtag der ønskes opsat
				
				\item \label{uc7sms} UC6: Ændre SMS-modtager udføres
																													
			\end{enumerate}\\\hline
											
		%\textbf{Version}		&1.0: 2014-03-23 Oprettet\\\hline
	\end{longtable}
	\label{UC7} 
\end{center}