Kommunikationen mellem PC og CSS hovedenheden sker over seriel kommunikation på et RS232 interface.

Det fysiske setup for RS232-interfacet er: 9600 kbps, ingen paritet, 8 bits, 1 stop bit.

I tabel \ref{table:RS232StartStopBytes} beskrives de fælles informationer som gælder mellem computeren og CSS hovedenheden.

\begin{table}[h]
	\caption{Start og stop bytes for RS232 kommunikation}
	\centering
	\begin{tabular}{|c|c|c|}
		\hline 
		& \textbf{ASCII} & \textbf{Hex} \\ 
		\hline 
		\textbf{STX} & 'S' / 's' & 0x53 / 0x73 \\ 
		\hline 
		\textbf{ETX} & 'cr' & 0x0D \\ 
		\hline 
	\end{tabular} 
	\label{table:RS232StartStopBytes}
\end{table}


Dataen formateres som vist i tabel \ref{table:RS232DataFormat}. <Data> blokken bruges kun i tilfælde af at der skal overføres en adresse.

\begin{table}[h]
	\caption{Data formatering for RS232 kommunikation}
	\centering
	\begin{tabular}{|l|c|c|c|c|}
		\hline 
		\textbf{Byte} & 0 & 1 & 2..5 & 2/6 \\ 
		\hline 
		\textbf{Indhold} & STX & <Kommando> & <Data> & ETX \\ 
		\hline 
	\end{tabular} 
	\label{table:RS232DataFormat}
\end{table}

\textbf{Blokken <Kommando>}

Kun kommandoerne beskrevet i tabel \ref{tabel:RS232Kommandoer} er gyldige. I tilfælde af at kommandoen ikke genkendes er der intet svar. Bemærk at det er muligt at bruge både store og små karakterer.

\begin{table}[h]
\caption{Kommandoer for RS232 kommunikation}
\centering
\begin{tabular}{|c|c|c|}
\hline 
\textbf{ASCII} & \textbf{HEX} & \textbf{Funktion} \\ 
\hline 
'A' / 'a' & 0x41 / 0x61 & Aktiver enhed \\ 
\hline 
'D' / 'd' & 0x44 / 0x64 & Deaktiver enhed \\ 
\hline 
'L' / 'l' & 0x4C / 0x6C & Hent login status \\ 
\hline 
'T' / 't' & 0x54 / 0x74 & Login korrekt \\ 
\hline 
'F' / 'f' & 0x46 / 0x66 & Login forkert \\ 
\hline
\end{tabular}
\label{tabel:RS232Kommandoer}
\end{table} 

\textbf{Aktiver- og Deaktiverkommandoerne}

For at bruge aktiver eller deaktiver kommandoerne er <Data> formateret som adressen. Denne adressering formateres som 4 byte, som hver består af ASCII karakterende '0' eller '1'. På den måde skriver man blot den adresse ind, som man har indstillet på sit X10 udtag. F.eks. "0100".
Bemærk at adresse "0000" ikke er gyldig.

\textbf{Eksempler:}

\textbf{"SA0101<cr>"}
Kommandoen aktiverer enheden med adresse "0101".

\textbf{"SL<cr>"}
Kommandoen beder CSS hovedenheden om at returnerer login status.

CSS Hovedenheden vil returnerer et svar:
\textbf{"ST<cr>"} for at brugeren er logget ind eller \textbf{"SF<cr>"} hvis brugeren ikke er logget ind.

