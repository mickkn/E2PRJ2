Kommunikationen mellem PC og CSS hovedenheden sker over seriel kommunikation på et RS232 interface.

Det fysiske setup for RS232-interfacet er: 9200 kbps, ingen paritet, 8 bits, 1 stop bit.

For alt kommunikation gælder.
\begin{tabular}{|c|c|c|}
\hline 
STX & 'S' / 's' & 0x53 / 0x73 \\ 
\hline 
ETX & 'cr' & 0x0D \\ 
\hline 
\end{tabular} 

Formateringen er som følger.

\begin{tabular}{|c|c|c|c|}
\hline 
STX & <Kommando> & <Data> & ETX \\ 
\hline 
\end{tabular} 

\textbf{<Kommando>}
Følgende kommandoer er gyldige.

\begin{tabular}{|c|c|c|}
\hline 
'A' / 'a' & 0x41 / 0x61 & Aktiver enhed \\ 
\hline 
'D' / 'd' & 0x44 / 0x64 & Deaktiver enhed \\ 
\hline 
'L' / 'l' & 0x4C / 0x6C & Hent login status \\ 
\hline 
'T' / 't' & 0x54 / 0x74 & Login korrekt \\ 
\hline 
'F' / 'f' & 0x46 / 0x66 & Login forkert \\ 
\hline
\end{tabular} 

\textbf{Aktiver og Deaktiver}
For at bruge aktiver eller deaktiver kommandoerne er <Data> formateret som adressen. Denne adressering formateres som 4 byte, som hver består af ASCII karakterende '0' eller '1'. På den måde skriver man blot den adresse ind, som man har indstillet på sit X10 udtag. F.eks. "0100".
Bemærk at adresse "0000" ikke er gyldig.

Eksempeler:
\textbf{"SA0101<cr>"}
Kommandoen aktiverer enheden med adresse "0101".

\textbf{"SL<cr>"}
Kommandoen beder CSS hovedenheden om at returnerer login status.

CSS Hovedenheden vil så returnerer et svar:
\textbf{"ST"} eller \textbf{"SF"}
T for at brugeren er logget ind eller F hvis brugeren ikke er logget ind.

