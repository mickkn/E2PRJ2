Kommunikationen mellem CSS Hovedenhed og X10 Udtagene sker over strømnettet via et X10 interface.

For alt kommunikation mellem CSS hovedenheden og X10 Udtagene gælder om STX (Start of text), start-byten, og ETX (End of text), slut-byten.

\begin{table}[h]
	\caption{Start og stop bytes for X10 kommunikation}
	\centering
	\begin{tabular}{|c|c|c|}
		\hline 
		& ASCII & Hex \\ 
		\hline 
		\textbf{STX} & 'S' / 's' & 0x53 / 0x73 \\ 
		\hline 
		\textbf{ETX} & '\textbackslash r' & 0x0D \\ 
		\hline 
	\end{tabular} 
	\label{table:X10StartStopBytes}
\end{table}

Data formateres som vist i tabel \ref{table:X10DataFormat}.

\begin{table}[h]
	\caption{Data formatering for X10 kommunikation}
	\centering
	\begin{tabular}{|r|c|c|c|c|}
		\hline 
		\textbf{Byte} & 0 & 1..4 & 5 & 6 \\ 
		\hline 
		\textbf{Indhold} & STX & <Adresse> & <Kommando> & ETX \\ 
		\hline 
	\end{tabular} 
	\label{table:X10DataFormat}
\end{table}

\textbf{<Kommando>}

Kun kommandoerne beskrevet i tabel \ref{tabel:X10Kommandoer} er gyldige. I tilfælde af at kommandoen ikke genkendes er der intet svar. Bemærk at det er muligt at bruge både store og små karakterer.

\begin{table}[h]
\caption{Kommandoer for X10 kommunikation}
\centering
\begin{tabular}{|c|c|c|}
\hline 
\textbf{ASCII} & \textbf{HEX} & \textbf{Funktion} \\ 
\hline 
'A' / 'a' & 0x41 / 0x61 & Aktiver enhed \\ 
\hline 
'D' / 'd' & 0x44 / 0x64 & Deaktiver enhed \\ 
\hline
\end{tabular}
\label{tabel:X10Kommandoer}
\end{table} 

\textbf{Aktiver- og Deaktiver kommandoerne}

For at bruge aktiver eller deaktiver kommandoerne er <Adresse> formateret som adressen. Denne adressering formateres som 4 byte, som hver består af ASCII karakterende '0' eller '1'. På den måde skriver man blot den adresse ind, som man har indstillet på sit X10 udtag. F.eks. "0100".
Bemærk at adresse "0000" ikke er gyldig.

\textbf{Eksempler:}

\textbf{"S0101A\textbackslash r"}
Kommandoen aktiverer X10 udtaget med adresse ''0101''.

\textbf{"s0101d\textbackslash r"}
Kommandoen deaktiverer X10 udtaget med adresse ''0101''.

Bemærk at \textbackslash r er ASCII karakteren for carriage return.

