%
% System arkitektur
%
\chapter{System Arkitektur}

\begin{figure}[htbp] \centering
\section{Domænemodel}
{\includegraphics[width=\textwidth]{billeder/diagrammer/Domain_Model}}
\caption{Domænemodel}
\label{lab:domainmodel}
\end{figure}
Domænemodel er udarbejdet i samarbejde med kunden. Denne har til opgave at give et struktureret billede af systemets funktionalitet og sammenhæng. Domænemodellen gør ikke brug af fagudtryk, men pile og kortfattede samt præcise sætninger anvendes for at beskrive sammenhængen mellem blokkene. Dette er med til at opnå en højere forståelse, af systemet som helhed, for kunden.

%% Protokol
\section{Protokol}

%%% Seriel kommunikation
\subsection{Seriel kommunikation}
Kommunikationen mellem PC og CSS hovedenheden sker over seriel kommunikation på et RS232 interface.

Det fysiske setup for RS232-interfacet er: 9200 kbps, ingen paritet, 8 bits, 1 stop bit.

For alt kommunikation gælder.
\begin{tabular}{|c|c|c|}
\hline 
STX & 'S' / 's' & 0x53 / 0x73 \\ 
\hline 
ETX & 'cr' & 0x0D \\ 
\hline 
\end{tabular} 

Formateringen er som følger.

\begin{tabular}{|c|c|c|c|}
\hline 
STX & <Kommando> & <Data> & ETX \\ 
\hline 
\end{tabular} 

\textbf{<Kommando>}
Følgende kommandoer er gyldige.

\begin{tabular}{|c|c|c|}
\hline 
'A' / 'a' & 0x41 / 0x61 & Aktiver enhed \\ 
\hline 
'D' / 'd' & 0x44 / 0x64 & Deaktiver enhed \\ 
\hline 
'L' / 'l' & 0x4C / 0x6C & Hent login status \\ 
\hline 
'T' / 't' & 0x54 / 0x74 & Login korrekt \\ 
\hline 
'F' / 'f' & 0x46 / 0x66 & Login forkert \\ 
\hline
\end{tabular} 

\textbf{Aktiver og Deaktiver}
For at bruge aktiver eller deaktiver kommandoerne er <Data> formateret som adressen. Denne adressering formateres som 4 byte, som hver består af ASCII karakterende '0' eller '1'. På den måde skriver man blot den adresse ind, som man har indstillet på sit X10 udtag. F.eks. "0100".
Bemærk at adresse "0000" ikke er gyldig.

Eksempeler:
\textbf{"SA0101<cr>"}
Kommandoen aktiverer enheden med adresse "0101".

\textbf{"SL<cr>"}
Kommandoen beder CSS hovedenheden om at returnerer login status.

CSS Hovedenheden vil så returnerer et svar:
\textbf{"ST"} eller \textbf{"SF"}
T for at brugeren er logget ind eller F hvis brugeren ikke er logget ind.



\newpage
%%% X10 kommunikation
\subsection{X10 kommunikation}
Kommunikationen mellem CSS Hovedenhed og X10 Udtagene sker over strømnettet via et X10 interface.

Den officielle X10 protokol bruges som udgangspunkt for denne arbitrerer X10 protokol.
Afvigelserne fra den officielle X10 protokol ligger i hvilke gyldige kommandoer der er til rådighed. Kun kommandoerne i tabel \ref{tabel:X10Kommandoer} er gyldige. Se også tabel \ref{table:X10DataFormat} for hvordan en data frame er bygget op. Bemærk at   enheds adressen og kommando sendes i én pakke.

\begin{table}[h]
	\caption{Data formatering for X10 kommunikation}
	\centering
	\begin{tabular}{|c|c|c|c|}
		\hline 
<<<<<<< HEAD
		STX & <Kommando> & <Adresse> & ETX \\
=======
		\textbf{ETX} & '\textbackslash r' & 0x0D \\ 
>>>>>>> parent of fe63449... Git hub fucker. committer ALT i repos.
		\hline 
	\end{tabular} 
	\label{table:X10DataFormat}
\end{table}

I det følgende differentieres der mellem almindelige binære mønstre og X10 formaterede bit mønstre.
Den officielle X10 protokol beskriver at det binære 0 sendes som 01 og binært 1 sendes som 10. Se \textbf{INDSÆT REFFERENCE TIL X10 PROTOKOLLEN} for yderligere detaljer.

I tabel \ref{table:X10StartStopBytes} beskrives de fælles informationer som gælder mellem CSS hovedenheden.

\begin{table}[h]
	\caption{Start og stop bytes for X10 kommunikation}
	\centering
	\begin{tabular}{|c|c|}
		\hline 
		& \textbf{X10 kode} \\ 
		\hline 
		\textbf{STX} & 1110\\ 
		\hline 
		\textbf{ETX} & 000000 \\ 
		\hline 
	\end{tabular} 
	\label{table:X10StartStopBytes}
\end{table}

\textbf{Blokken <Kommando>}

Alle kommandoer sendes to gange med tre 50 Hz perioder i mellem hver. Dette håndteres med ETX koden.
I tilfælde af at kommandoen ikke genkendes er der intet svar.

\begin{table}[h]
\caption{Kommandoer for X10 kommunikation}
\centering
\begin{tabular}{|c|c|c|}
\hline 
\textbf{Binær} & \textbf{X10 kode} & \textbf{Funktion} \\ 
\hline 
00101 & 0101100110 & Aktiver enhed \\ 
\hline 
00111 & 0101101010 & Deaktiver enhed \\ 
\hline
\end{tabular}
\label{tabel:X10Kommandoer}
\end{table} 

\textbf{Blokken <Adresse>}

Adresserne modtages fra PCen binært. Denne kode omsættes til X10 formatet og afsendes uden videre formatering.
I tabel \ref{tabel:X10Adresser} vises nogle eksempler på adresser.

\begin{table}[h]
\caption{Adresser formateret i X10 format}
\centering
\begin{tabular}{|c|c|}
\hline 
\textbf{Binær} & \textbf{X10 kode} \\ 
\hline 
0001 & 01010110 \\ 
\hline 
0101 & 01100110 \\ 
\hline
0111 & 01101010 \\
\hline
\end{tabular}
\label{tabel:X10Adresser}
\end{table} 

<<<<<<< HEAD
\textbf{Eksempler}

I tabel \ref{tabel:X10Eksempler} er vist to eksempler som aktiverer og deaktiverer et X10 udtag. Mellemrummende i X10 koden er indsat for at kunne se blokkende og vil ikke eksisterer i praksis.

\begin{table}[h]
\caption{Adresser formateret i X10 format}
\centering
\begin{tabular}{|c|c|c|}
\hline 
\textbf{Kommando} & \textbf{X10 kode} \\ 
\hline 
Tænd X10 udtag på adresse 0101 & 1110 0101100110 01100110 000000 \\ 
\hline 
Sluk X10 udtag på adresse 0011 & 1110 0101101010 01011010 000000 \\ 
\hline
\end{tabular}
\label{tabel:X10Eksempler}
\end{table} 
=======
\textbf{"S0101A\textbackslash r"}
Kommandoen aktiverer X10 udtaget med adresse ''0101''.

\textbf{"s0101d\textbackslash r"}
Kommandoen deaktiverer X10 udtaget med adresse ''0101''.

Bemærk at \textbackslash r er ASCII karakteren for carriage return.

>>>>>>> parent of fe63449... Git hub fucker. committer ALT i repos.


\newpage
%% Hardware
\section{Hardware}

\subsection{Hardware beskrivelse}
\begin{figure}[!htbp] \centering
\subsection{BDD Hardware}
{\includegraphics[width=0.9\textwidth]{billeder/diagrammer/BDD_Hardware}}
\caption{BDD Hardware}
\label{lab:bddhardware}
\raggedright
\end{figure}
BDD diagrammet giver et overblik over hvad det samlede system består af. Vi ser en port beskrivelse som viser hvilke signaler hver blok består af.

\begin{figure}[!htbp] \centering
\subsection{BDD Hovedenhed}
{\includegraphics[width=0.9\textwidth]{billeder/diagrammer/BDD_Hovedenhed}}
\caption{BDD Hovedenhed}
\label{lab:bddhovedenhed}
\raggedright
\end{figure}
BDD diagrammet giver et overblik over hvad CSS hovedenheden består af. Vi ser en portbeskrivelse for STK kittet og encoder samt et overblik over hvilke komponenter encoderen består af. 

\begin{figure}[!htbp] \centering
\subsection{BDD Modtager}
{\includegraphics[width=0.9\textwidth]{billeder/diagrammer/BDD_Modtager}}
\caption{BDD Modtager}
\label{lab:bddmodtager}
\raggedright
\end{figure}
BDD diagrammet giver et overblik over hvad X10 modtageren består af. Vi ser en portbeskrivelse for STK kittet og decoderen samt et overblik over hvilke komponenter decoderen består af.

\begin{figure}[!htbp] \centering
\subsection{IBD Hardware}
{\includegraphics[width=0.9\textwidth]{billeder/diagrammer/IBD_Hardware}}
\caption{IBD Hardware}
\label{lab:ibdhardware}
\raggedright
\end{figure}
IBD diagrammet giver et internt overblik over hvordan hele vores system er forbundet. Vi ser hvilke type signaler der bliver sendt imellem vores forskellige blokke.

\begin{figure}[!htbp] \centering
\subsection{IBD Hovedenhed og Modtager}
{\includegraphics[width=0.9\textwidth]{billeder/diagrammer/IBD_Hovedenhed_Modtager}}
\caption{IBD Hovedenhed og Modtager}
\label{lab:ibdhovedenhedmodtager}
\raggedright
\end{figure}
IBD diagrammet giver et internt overblik over hvordan vores X10 hovedenhed og X10 modtager er forbundet. Vi ser hvilke type signaler der bliver sendt imellem vores forskellige blokke.


\clearpage
\newpage

\begin{table}[htbp] %% Blok og Signal Tabel
\subsection{Grænseflade}
For at opnå forståelse for signaler mellem blokkene laves en grænseflade der beskriver de enkelte blokkes porte og hvilke signaler der løber mellem disse.

\subsubsection{Blok beskrivelse}
Til at beskrive blokkene nærmere er anvendt tabeller som ses herunder. Her er hvert signal i en respektiv blok kommenteret og blokkens funktion er kort beskrevet. 

\caption{Tabel med beskrivelse af respektive blokke}
\begin{small}
\begin{tabular}{|p{3,3cm}|p{3,3cm}|p{3,3cm}|p{3,3cm}|}
\hline
\textbf{Bloknavn} & \textbf{Funktion} & \textbf{Signaler} & \textbf{Kommentar} \\ \hline

Encoder & modtage kommando og encode til 120 kHz bursts & 120 kHz & Data ud \\ \cline{3-4}	
& & X10 data & X10 data kommando ind \\ \hline

STK500 Hovedenhed & Genererer burst og detekterer på zero-crossing & RS232 & Laptop forbindelse \\ \cline{3-4}
& & X10 data & X10 data kommando linje \\ \cline{3-4}
& & Bool & lyd detektion \\ \cline{3-4}
& & Bool & Password accept \\ \hline

Decoder & Modtager 120 kHz og decoder til X10 data & 120 kHz & 120 kHz ind \\ \cline{3-4}
& & X10 data & Kommando linje \\ \cline{3-4}
& & 120 kHz & 120 kHz data ind \\ \hline

STK 500 Decoder & Modtager burst og detekterer på zero-crossing & X10 data & X10 data ind \\ \cline{3-4}
&& Bool & Power I/O ekstern enhed \\ \hline 
\end{tabular}
\end{small}
\label{table:Bloktabel}
\end{table}

\begin{table}[htbp]
\subsubsection{Signal beskrivelse}
For at fuldende beskrivelsen af grænsefladen er der lavet en signaltabel som kan ses herunder. Hvert signal er beskrevet og tilknyttet en kort kommentar. Området et signal er defineret under er også beskrevet. Blok og terminal indgår også. 
\caption{Tabel over signaler med terminaler}
\begin{small}
\begin{tabular}{|p{2cm}|p{2cm}|p{2cm}|p{2cm}|p{2cm}|p{2,2cm}|}
\hline
\textbf{Signal-navn} & \textbf{Funktion} & \textbf{Område} & \textbf{Port 1} & \textbf{Port 2} & \textbf{Kommentar} \\ \hline

120 kHz & sende kommando på 18V nettet & & Encoder, HE2 & Decoder, DM1 & \\ \hline

X10 data & kommando & & STK500, HSTK4 & Encoder, HE1 & \\ \cline{4-5}
&&& Decoder, MD2 & STK500, MSTK1 &\\ \hline

bool & digital signal & 5V/0V & DE2, N/A & STK500, HSTK2 & \\ \cline{4-5}
&&& Lyddetektor, N/A & STK500, HSTK3 & \\ \cline{4-5}
&&& STK500, MSTK2 & Udtag, N/A & \\ \hline
\end{tabular}
\end{small}
\label{table:Signaltabel}
\end{table}




\newpage
%% Software
\section{Software}

%%% Applikationsmodel for PC
\subsection{Applikations model for PC}
\input{filer/UML/applicationsmodel_pc}
\clearpage

%%%% Klasse beskrivelser
\subsection{Klassebeskrivelse for PC}
Her følger klassebeskrivelser for alle klasser til PC. \\

{\centering
\textbf{Hukommelse klasse}\par
}

void saveLogin(bool); \\
\textbf{Parametre:} ingen \\
\textbf{Returværdi:} ingen \\
\textbf{Beskrivelse:} gemmer login status og bevare denne i 10 minutter \\

void saveStatus(bool, int adresse); \\
\textbf{Parametre:} bool til bestemmelse af om status er aktiv eller deaktiv. Int adresse til bestemmelse af status på adressen \\
\textbf{Returværdi:} ingen \\
\textbf{Beskrivelse:} gemmer status på enheden på pågældende adresse \\

void getEnheder); \\
\textbf{Parametre:} ingen \\
\textbf{Returværdi:} ingen \\
\textbf{Beskrivelse:} gemmer login status og bevare denne i 10 minutter \\

int getNumber(); \\
\textbf{Parametre:} ingen \\
\textbf{Returværdi:} gemte telefonnummer \\
\textbf{Beskrivelse:} returnere det gemte telefonnummer \\

void saveNumber(int number); \\
\textbf{Parametre:} number der skal gemmes \\
\textbf{Returværdi:} ingen \\
\textbf{Beskrivelse:} gemmer telefonnummeret \\

{\centering 
\textbf{Login klasse}\par
}

void loginValid(); \\
\textbf{Parametre:} ingen  \\
\textbf{Returværdi:} ingen \\
\textbf{Beskrivelse:} N/A \\

{\centering 
\textbf{Aktiver klasse}\par
}

void aktiverEnhed(int adresse); \\
\textbf{Parametre:} adresse på enhed \\
\textbf{Returværdi:} ingen \\
\textbf{Beskrivelse:} N/A \\

{\centering 
\textbf{Deaktiver klasse}\par
}

void deaktiverEnhed(int adresse); \\
\textbf{Parametre:} adresse på enhed \\
\textbf{Returværdi:} ingen \\
\textbf{Beskrivelse:} N/A \\

{\centering 
\textbf{DetekterLyd klasse}\par
}

void lydDetekteret(); \\
\textbf{Parametre:} ingen \\
\textbf{Returværdi:} ingen \\
\textbf{Beskrivelse:} henter telefonnummer i hukommelse og sender det til ClickATell klassen \\

{\centering 
\textbf{RedigerSmSBruger klasse}\par
}

void redigerSMS(int number); \\
\textbf{Parametre:} nye nummer \\
\textbf{Returværdi:} ingen \\
\textbf{Beskrivelse:} gemmer nye nummer i hukommelsen \\

{\centering 
\textbf{Udtag klasse}\par
}

bool addUdtag(int adresse, string name); \\
\textbf{Parametre:} adresse og navn på udtag \\
\textbf{Returværdi:} true hvis operation gik godt, false hvis ikke \\
\textbf{Beskrivelse:} tilføjer udtag til hukommelse ved at gemme navn og adresse \\

{\centering 
\textbf{ClickATellIF klasse}\par
}

void sendSMS(int number); \\
\textbf{Parametre:} telefonnummer \\
\textbf{Returværdi:} ingen \\
\textbf{Beskrivelse:} sender sms til bruger via clickatell \\

{\centering 
\textbf{RS232IF klasse}\par
}

bool loginValid(); \\
\textbf{Parametre:} ingen \\
\textbf{Returværdi:} true eller false \\
\textbf{Beskrivelse:} afventer login fra DE2 board \\

void aktiver(int adresse); \\
\textbf{Parametre:} adresse på enhed \\
\textbf{Returværdi:} ingen \\
\textbf{Beskrivelse:} beder om aktivering af enhed på adressen, ifølge protokol \\

void deaktiver(int adresse); \\
\textbf{Parametre:} adresse på enhed \\
\textbf{Returværdi:} ingen \\
\textbf{Beskrivelse:} beder om deaktiver af enhed på adressen, ifølge protokol \\

{\centering 
\textbf{BrugerUI klasse}\par
}

void showMenu(); \\
\textbf{Parametre:} ingen \\
\textbf{Returværdi:} ingen \\
\textbf{Beskrivelse:} skal styre hele brugerUI menuen. \\

bool printStatus(); \\
\textbf{Parametre:} ingen \\
\textbf{Returværdi}: bool godkendt \\
\textbf{Beskrivelse:} hente status og navne fra hukommelse og udskrive dem på skærmen \\













\newpage

%%% Applikationsmodel for CSS hovedenhed
\subsection{Applikations model for CSS hovedenhed}
\input{filer/UML/applicationsmodel_css_hovedenhed}
\clearpage

%%%% Klasse diagram og beskrivelser
\subsection{Klassediagram og beskrivelse for CSS hovednehed}
Ud fra applikationsmodellens klasse diagram (Figur \ref{fig:CSS_hovedenhed_Class}) er udledt et statisk klasse diagram, se figur \ref{fig:CSS_hovedenhed_Class_Static}.
\begin{figure}[!htb] \centering
     \includegraphics[width=\textwidth]{billeder/uml/CSS_hovedenhed_Class_Static}
     \caption{Statisk klassediagram for CSS hovedenhed}
     \label{fig:CSS_hovedenhed_Class_Static}
\end{figure}

Her følger klassebeskrivelser for alle klasser til CSS hovedenheden. \\

%
% RS232IF
%
{\centering
\textbf{RS232IF}\par
}
\textbf{Ansvar:} At varetage kommunikation mellem CSS hovedenhed og PC over RS232 protokollen. \\
\textbf{Attributer:}
\begin{itemize}
	\item \textbf{UC1Login * UC1Ptr\_} \\
	Pointer til associeret UC1 objekt
	\item \textbf{UC2Aktiver * UC2Ptr\_} \\
	Pointer til associeret UC2 objekt
	\item \textbf{UC3Deaktiver * UC3Ptr\_} \\
	Pointer til associeret UC3 objekt
\end{itemize}
\textbf{Metoder:} \\
void adviser(); \\
\textbf{Parametre:} Ingen \\
\textbf{Returværdi:} Ingen \\
\textbf{Beskrivelse:} Sender kommando "SB<cr>" over RS232 \\

%
% UC1Login
%
{\centering
\textbf{UC1Login}\par
}
\textbf{Ansvar:} At varetage UC1 Login forløbet. \\
\textbf{Attributer:}
\begin{itemize}
	\item \textbf{DE2IF * DE2Ptr\_} \\
	Pointer til associeret DE2 objekt
\end{itemize}
\textbf{Metoder:} \\
bool loginValid(); \\
\textbf{Parametre:} Ingen \\
\textbf{Returværdi:} Ingen \\
\textbf{Beskrivelse:} Kalder getLoginStatus() metoden i DE2IF og returnerer værdien her fra \\

%
% UC2Aktiver
%
{\centering
\textbf{UC2Aktiver}\par
}
\textbf{Ansvar:} At varetage UC2 Aktiver forløbet. \\
\textbf{Attributer:}
\begin{itemize}
	\item \textbf{X10udtagIF * X10Ptr\_} \\
	Pointer til associeret X10udtag objekt
\end{itemize}
\textbf{Metoder:} \\
void aktiver(int adresse); \\
\textbf{Parametre:} Adresse på enhed \\
\textbf{Returværdi:} Ingen \\
\textbf{Beskrivelse:} Kalder aktiver() metoden i X10udtagIF med den modtagede adresse \\

%
% UC3Deaktiver
%
{\centering
\textbf{UC3Deaktiver}\par
}
\textbf{Ansvar:} At varetage UC3 Deaktiver forløbet. \\
\textbf{Attributer:}
\begin{itemize}
	\item \textbf{X10udtagIF * X10Ptr\_} \\
	Pointer til associeret X10udtag objekt
\end{itemize}
\textbf{Metoder:} \\
void deaktiver(int adresse); \\
\textbf{Parametre:} Adresse på enhed \\
\textbf{Returværdi:} Ingen \\
\textbf{Beskrivelse:} Kalder deaktiver() metoden i X10udtagIF med den modtagede adresse \\

%
% UC5DetekterLyd
%
{\centering
\textbf{UC5DetekterLyd}\par
}
\textbf{Ansvar:} At varetage UC5 Detekter Lyd. \\
\textbf{Attributer:}
\begin{itemize}
	\item \textbf{RS232IF * RS232Ptr\_} \\
	Pointer til associeret X10udtag objekt
\end{itemize}
\textbf{Metoder:} \\
void detekterLyd(); \\
\textbf{Parametre:} Ingen \\
\textbf{Returværdi:} Ingen \\
\textbf{Beskrivelse:} Kalder adviser() metoden i RS232IF \\

%
% DE2IF
%
{\centering
\textbf{DE2IF}\par
}
\textbf{Ansvar:} At holde styr på aktuel loginstatus på DE2 boardet. \\
\textbf{Attributer:}
\begin{itemize}
	\item \textbf{bool loginStatus\_} \\
	1: Login bekræftet på DE2 board \\
	0: Login ikke bekræftet på DE2 board
\end{itemize}

\textbf{Metoder:} \\
void setLoginStatus(bool status); \\
\textbf{Parametre:} status: 1 hvis bekræftet og 0 hvis ikke bekræftet på DE2 board \\
\textbf{Returværdi:} Ingen \\
\textbf{Beskrivelse:} Sætter attribut loginStatus\_ til aktuel status på DE2 board \\

bool getLoginStatus(); \\
\textbf{Parametre:} Ingen \\
\textbf{Returværdi:} status: 1 hvis bekræftet og 0 hvis ikke bekræftet på DE2 board \\
\textbf{Beskrivelse:} Returnerer aktuel login status \\

%
% X10udtagIF
%
{\centering
\textbf{X10udtagIF}\par
}
\textbf{Ansvar:} At varetage kommunikation mellem CSS hovedenhed og X10 modtager over X10 protokollen. \\
\textbf{Attributer:} Ingen \\
\textbf{Metoder:} \\
void aktiver(int adresse); \\
\textbf{Parametre:} adresse: Adresse på X10 enhed som ønskes aktiveret \\
\textbf{Returværdi:} Ingen \\
\textbf{Beskrivelse:} Konverterer adressen bitvis til ASCII karrakterer (0010 -> ''0010'') og sender kommandoen "SAXXXX<cr>" (hvor XXXX er adressen over) over X10 \\

void deaktiver(int adresse); \\
\textbf{Parametre:} adresse: Adresse på X10 enhed som ønskes deaktiveret \\
\textbf{Returværdi:} Ingen \\
\textbf{Beskrivelse:} Konverterer adressen bitvis til ASCII karrakterer (0010 -> ''0010'') og sender kommandoen "SDXXXX<cr>" (hvor XXXX er adressen over) X10 \\

%
% SensorIF
%
{\centering
\textbf{SensorIF}\par
}
\textbf{Ansvar:} At holde styr på aktuel lyd detektering og give besked hvis lyd registreres. \\
\textbf{Attributer:}
\begin{itemize}
	\item \textbf{bool loginStatus\_} \\
	1: Lyd detekteret \\
	0: Ingen lyd detekteret
	\item \textbf{UC5DetekterLyd * UC5Ptr\_} \\
	Pointer til associeret UC5 objekt
\end{itemize}

\textbf{Metoder:} \\
void setLydNiveau(bool niveau); \\
\textbf{Parametre:} niveau: 1 hvis lyd detekteret ellers 0 \\
\textbf{Returværdi:} Ingen \\
\textbf{Beskrivelse:} Kalder detekterLyd() metoden, i UC5DetekterLyd, med parameter 1, hvis niveau er 1, ellers intet. \\


\newpage

\newpage
%%% Applikations model for X10 modtager
\subsection{Applications model for X10 modtager}
\input{filer/UML/applicationsmodel_x10_modtager}
\clearpage

%%%% Klasse beskrivelser
\subsection{Klassebeskrivels for X10 modtager}
%
% UC1Login
%
{\centering
\textbf{UC1Login}\par
}
\textbf{Ansvar:} At varetage UC1 Login forløbet. \\
\textbf{Attributer:}
\begin{itemize}
	\item \textbf{DE2IF * DE2Ptr\_} \\
	Pointer til associeret DE2 objekt
\end{itemize}
\textbf{Metoder:} \\
bool loginValid(); \\
\textbf{Parametre:} Ingen \\
\textbf{Returværdi:} Ingen \\
\textbf{Beskrivelse:} Kalder getLoginStatus() metoden i DE2IF og returnerer værdien her fra \\

\newpage