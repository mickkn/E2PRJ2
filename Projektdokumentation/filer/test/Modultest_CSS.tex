Alle klasser i CSS hovedenhedspakken er testet. De enkelte tests er beskrevet i det følgende afsnit.
Se det statiske klassediagram for at se sammenhængen mellem klasserne i figur \ref{fig:CSS_hovedenhed_Class_Static}.

% Timer
\subsection{Timer}
Timer klassen er testet med programmet "Timer Test.cpp" som ligger i kildekoden til klassen.
Programmet opretter et objekt af Timer typen og starter et endeløst loop som tænder timeren i 1 sekund og slukker den i 1 sekund.

\textbf{Opstilling}

En oscilloscope probe sættes på PD5 på et STK500 kit som har programmet kørende.


%På figur \ref{fig:Test_CSS_Timer_scope} vises et oscilloscope billede målt på udgangen af timeren PD5.

%\begin{figure}[!htb]
%     {\includegraphics[width=\textwidth]{billeder/SWTest/CSS_Timer_scope}}
%     \caption{Oscilloscope måling på udgangen af Timer1 på STK500 kit.}
%     \label{fig:Test_CSS_Timer_scope}
%\end{figure}

% UART
\subsection{UART}
UART klassen er testet med programmet "UART Test.cpp" som ligger i kildekoden til klassen.
Programmet starter med at sende en streng over RS232 interfacet. Her efter afventer den at modtage en hel kommando. Når denne er modtaget sendes den retur.

\textbf{Opstilling}

Test programmet ligges på et STK500 kit.
En computer forbindes med STK500 kittet over RS232 Spare porten (Sub-D).
En jumper forbinder headeren RXD med PD0 og TXD med PD1.
Start et terminalprogram op indstillet til 9600 buad, 8 databit, 1 stopbit og ingen paritet.
Reset STK500. Først modtages strengen ''CSS hovedenhed\textbackslash r\textbackslash n''.
Send igennem terminal her efter, én karakter af gangen: ''A0101\textbackslash r''. Dette modtages igen som svar.
Send igen, én karakter af gange: ''D0011\textbackslash r''. Dette modtages igen som svar.

% X10IF
\subsection{X10IF}


% RS232IF
\subsection{RS232IF}

% CircBuffer
\subsection{CircBuffer}

% ZeroCrossInt
\subsection{ZeroCrossInt}
