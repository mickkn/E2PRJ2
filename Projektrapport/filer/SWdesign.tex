% SW design
\section{Software design}

Med udgangspunkt i domænemodellen udviklet i arkitekturfasen er der udviklet applikationsmodeller for hver computer i systemet. Dette giver overblik over de funktionaliteter som skal implementeres på de forskellige platforme.

Applikationsmodellen består af at beskrive hvordan information fordeles i hvert UC. Dette opnåes med tre diagram typer. Sekvensdiagrammer som viser hvordan information bevæger sig sekventielt igennem systemets klasser, et klassediagram som sammenfatter de metoder og relationer som er fundet i sekvensdiagrammet og et tilstandsmaskinediagram som viser et systems forskellige tilstande. Det sidste er udeladt da det ikke er aktuelt for det opbyggede system.

I det følgende vises applikationsmodellen for X10 udtaget. For modeller over CSS hovedenhed og PC henvises til projektdokumentationen.

\subsection{Applikationsmodel for X10 udtag}
Først er der lavet en detaljeret domænemodel for X10 udtaget. Denne er vist i figur \ref{fig:X10_udtag_domaenemodel}. Denne laves ved at gennemgå UC beskrivelserne og finde de ting som har indflydelse på netop denne del af systemet.

\begin{figure}[!htb]
     \centering
     { \includegraphics{Billeder/UML/X10_modtager_Domain}}
     \caption{Domænemodel for X10 udtag}
     \label{fig:X10_udtag_domaenemodel}
\end{figure}

Med dette udgangspunkt laves der sekvensdiagrammer for hvert UC. Disse er vist i figur \ref{fig:X10_udtag_sd}. De viser hvordan metodekald i mellem de konceptuelle klasser og giver et overblik over den basale funktionalitet.

\begin{figure}[!htb]
     \centering
     { \includegraphics{Billeder/UML/X10_modtager_SD}}
     \caption{Sekvensdiagram for X10 udtag}
     \label{fig:X10_udtag_sd}
\end{figure}

Dette resulterer i et klassediagram med grundfunktionaliteten beskrevet, se figure \ref{fig:X10_udtag_class}. Denne bruges under implementeringen og ender ud i et statisk klassediagram som beskriver det endelige program med alle hjælpemetoder.

\begin{figure}[!htb]
     \centering
     { \includegraphics{Billeder/UML/X10_modtager_Class}}
     \caption{Klassediagram for X10 udtag}
     \label{fig:X10_udtag_class}
\end{figure}

Denne analyse af funktionalitet giver et klart overblik til implementeringsfasen.