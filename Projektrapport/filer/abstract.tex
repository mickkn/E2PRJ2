\chapter*{Abstract (BS)}
The following document describes the work and process of the groups 2nd term project at Aarhus School of Engineering. The purpose of the project is to use and evaluate the methods and subjects taught at this terms courses. The report describes how the product came from an idea to a physical product as well as the details of the product and the methods used.

The product developed is a child security system protecting unattended children from electrical shock and heat burns as a course of dangerous home appliances. From a computer program the user can turn on and off the power of mains outlets without the other cables than that of the mains. A code lock secures that only authorised users can access to program and that way parents can disable dangerous home appliances when they leave their children unattended.

Development is managed with the ASE-model, which is a semi-iterative project management process. Further more is it done on different known platforms introduces first and second term. As transmitting and receiving units Atmel’s STK500 development board is used along with specialised hardware. A PC is used as the user interface and the DE2 board from Altera is used as the code lock.

The results include a functional user interface, an X10 transmitter and an almost functional X10 receiver.