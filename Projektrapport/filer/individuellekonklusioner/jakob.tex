\section{Jakob Schmidt}
Sammenlignet med 1. semester projektet, så har processen i det her projekt været meget mere struktureret. Sammenholdet og overblikket mellem hvert projektmedlem har været rigtig godt. Det gode samarbejde og overblik skyldes at vi jævnligt har holdt møder, både som selvstændig gruppe, men også sammen med vejleder. Desuden er der afholdt enkelte trivselsrunder, hvor vi hver i sær har skulle fortælle hvordan vi selv følte det gik i projektet. 

Inden vi fik det endelige overblik over hvor omfattende vi kunne lave vores projekt, var det tydeligt at vi havde lidt for stor ambitioner til projektes omfang. Dette blev skåret ned efter første reviewmøde da vi fra anden vejleder blev anbefalet på det kraftigste at revurdere vores use cases.

Personligt har jeg beskæftiget mig med elektroniken i projektet, det kom som narturligt valg da jeg læser på elektro linjen. Fagligt har der været nogle komplikationer med at få de enkelte moduler til at virke som tiltænkt. Projektet har været spændende men samtidig udfordrende, specielt eftersom den nødvendige teori først var helt på plads i den sidste fase af forløbet.  

Overordnet set er jeg rigtig tilfreds med resultatet af vores projekt, på trods af at vi langt fra fik realiseret alle de ting vi havde udtænkt fra første brainstormmøde. Efter revurdering af projektets omfang er alle dele desværre stadig blevet færdige, men ideen er der og de vigtigste dele blev implementeret. 