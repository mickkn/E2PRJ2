\section{Mick Kirkegaard}
Vi besluttede i starten af projektfasen at bruge skriveredskabet LaTeX, og jeg blev lidt 
uventet primus-motor i dette værktøj. Dette har sammen med SVN-værktøjet GitHub givet en 
del hovedpine. Men jeg er rigtig glad for at være kommet igennem processen med disse værktøjer.

Selve projektet fik vi sparket igang med en fantastisk brainstorm med en masse gode ideér
som vi hurtigt fandt ud af, var alt for optimistisk. Der måtte derfor hurtigt skrues
ned for ideérne og fokuseres på hovedindholdet i projektet.

Jeg syntes processen har været rigtig sjov, det har været spændende at bruge de værktøjer
man lærer på studiet til noget reelt elektronik. Bare det at forstå X10 kommunikationens
virkemåde har været rigtigt spændende for mig som Elektro studerende. Dette var en af nødderne
der skulle knækkes før projektet virkelig blev spændende.

I hardwaregruppen har vi haft et tæt samarbejde og virkelig vendt og drejet alle aspekterne
i projektet med hinanden. Og med vores ugentlige møder, har vi kunnet følge lidt med i 
software gruppens arbejde med projektet. Alt-i-alt et godt forløb. Savner dog et karakter-givende
projekt næste gang, så man kan ligge endnu mere sjæl i det.