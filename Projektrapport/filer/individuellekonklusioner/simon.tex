\section{Simon Kirchheiner}
% Din konklusion her
Vi har været syv mand i vores projektgruppe, og det har givet os mulighed for at øve os på at samarbejde med flere om et projekt. Vi har været en god projektgruppe med mange forskellige kompetencer, det har medført at man kan hjælpe hinanden godt. Det var en stor fordel at hele projektet var opdelt i faser og havde løbende møder og reviews, så folk ikke bare kørte på helt selv. Faserne har hjulpet med at få et overblik over projektet, og gjort at vi har tænkt over de ting vi skulle lave. I starten af projektet havde vi store ambitioner om at nå mange forskellige ting, men som tiden skred frem måtte vi se i øjnene at det ikke var realistisk. 

I forhold til 1. semester har vi brugt mere tid på at dokumentere det vi skulle lave, og fået en forståelse for de forskellige arbejdsmetoder vi har lært i ISE. Jeg har været på HW delen og der har ASB/MSA fagene givet den nødvendige viden til at kunne løse de problemstillinger vi stod overfor. Jeg tog ansvaret for at skrive logbog/referater, det har givet et godt grundlag for at få struktureret arbejde, da alle kan gå ind i logbogen/referatet og se hvad vi snakkede om.

Vi fik et produkt næsten som planlagt, grundet tidsmangel fik vi ikke lavet vores lyddetektor, men vi fik kommunikation over AC nettet til at fungere korrekt. Jeg har fået et stort udbytte ud af dette projekt, og jeg synes vi har haft et godt samarbejde i projektgruppen. Generelt har gruppen fungeret godt og jeg mener vi har et rigtig godt produkt.
