\chapter{Indledning (BS, JS)}

Med udgangspunkt i børnesikkerhed i hjemmet er der blevet udviklet et produkt, som kan hjælpe familier med børn, til at få et mere sikkert hjem.

Af problemstillinger som kan opstå i en almindelig husholdning kan nævnes:
\begin{itemize}
	\item Fare for at et barn tænder for en kogeplade, eller andre elektriske varme aggregater, og efterfølgende kan brænde sig
	\item Fare for at et barn kan skære sig på køkkenknive som ligger i en skuffe
\end{itemize}

Den anden del af systemet er en babyalarm. Næsten alle mennesker i Danmark har deres mobiltelefon i nærheden hele tiden, så i stedet for at skulle have en babyalarm med rundt også, så kan man koble sin mobil til systemet og få besked når barnet giver lyd fra sig.

Dette ender ud i tre produkter:

\begin{itemize}
\item Afbryder til valgt 230 Vac stikkontakt
\subitem Beskyttelse mod kogeplader og lignende
\item Låsemekanisme til at låse skabe og skuffer
\subitem Aflåsning af skuffe med køkkenknive
\item Babyalarm til lyddetektering
\subitem SMS-beskeder i stedet for en ekstra ''boks'' i lommen
\end{itemize}

Systemet skal være nemt at sætte op og skal kommunikere over det eksisterende 230 V vekselspændings netværk i hus installationen.

En central computer håndterer styringen mellem enhederne og systemet kan aktiveres med et kodetryk.

For at udarbejde det endelige produkt er vi startet i fællesskab med at få lavet de nødvendige dokumentationer der skal bruges for at kunne udvikle sådan et produkt. Herunder er bl.a. use case diagram, kravspecifikation og accepttest. Herefter har gruppen delt sig i et hardware og software team. Hardware har taget til opgave at få udviklet de nødvendige elektroniske komponenter mens software har skrevet koden.

Begge teams har stået for den respektive dokumentation for det de fik udviklet gennem forløbet. 