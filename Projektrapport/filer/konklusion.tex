\chapter{Konklusion (PO, SK)}
%Konklusion

Formålet med projektet har været at konstruere et system for at øge børns sikkerhed i hjemmet. Det skulle være muligt at styre bestemt 230Vac udtag i hjemmet, og herigennem låse f.eks skuffer eller kogeplader. Ydermere ønskede vi at modernisere den udbredte babyalarm til en mobil løsning. 

Vi har opnået en prototype, som kan afsende en X10 kommando. PC softwaren, samt encoder softwaren virker som tiltænkt. Vi har her til slut ikke fået decoder softwaren til at virke sammen med decoder hardwaren. Vi oplever at den første bit ikke kommer på det tidspunkt den skal, mens resten af bitrækken kommer som ønsket. Dette kan skyldes en langsom opstart i hardwaren eller en software fejl. Prototypen er opbygget på veroboard, dette valg blev taget, da vi oplevede for mange løse forbindelser i fumlebrædderne. 
Det er lykkedes at få systemet til at afsende SMS.  

Samarbejdet har fungeret godt og ud fra Gruppekontrakten har gruppen haft et fælles standpunkt og grundlag for projektgennemførelsen. Gennem ASE-modellen er gruppen delt i 2 teams, der har haft hhv. hardware og software som fokuspunkt. Dette samarbejde har fungeret godt, både i de to teams samt gruppen samlet. Gruppen har arbejdet struktureret og professionelt med projektet. Gruppemøderne har spillet en væsentlig rolle for at opretholde strukturen.  

Vi har igennem \LaTeX oplevet opstartsvanskeligheder, men tilslut er vi rigtigt glade for at have valgt den løsning. 