\chapter*{Resumé (BS)}

Den følgende rapport beskriver arbejdet med gruppens semester-projekt for 2. semester på Aarhus School of Engineering. Formålet med projektet er at afprøve de metoder og emner som semesterets fag har introduceret. Rapporten beskriver arbejdsmetoderne der er brugt fra idé til produkt, det egentlige produkt og udviklingen af dette samt de overvejelser og værktøjer som er benyttet.

Projektet tager udgangspunkt i et system til at beskytte børn i hjemmet mod stød og forbrændinger fra elektriske apparater. Fra et computer program kan en bruger tænde og slukke enheder som er koblet til 230 Vac el-nettet uden behov for ekstra kabler. En kodelås forhindrer børnene i selv at tænde for strømmen igen og forældre kan dermed slukke for farlige elektriske apparater centralt, så børn ikke kommer til skade hvis de ikke er under opsyn. 

Udviklingsforløbet er styret efter ASE-modellen som er en halv-iterativt projektledelses-metode. Produktet er udviklet på de platforme der er gjort brug af på 1. og 2. semester. Som enheder til modtagelse og afsendelse af data over el-nettet bruges STK500 kittet fra Atmel samt egen udviklet hardware, til at interface til 230 Vac el-nettet og en almindelig PC bruges som bruger interface. Som kodelås anvendes DE2 boardet fra Altera.

Projektet er endt ud i et funktionelt bruger interface, en X10 afsender enhed og en næsten funktionel X10 modtager.
