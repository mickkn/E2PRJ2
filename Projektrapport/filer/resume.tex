\chapter{Resumé}

Den følgende rapport beskriver arbejdet med gruppens semesterprojekt på 2. semester på Aarhus School of Ingineering. Formålet med projektet er at afprøve de metoder og emner som semesterets fag har introduceret. Rapporten beskriver arbejdsmetoderne der er brugt fra idé til produkt, det egentlige produkt og udviklingen af dette samt de overvejelser og værktøjer som er benyttet.

Projektet tager udgangspunkt i et system til at beskytte børn i hjemmet. Fra et computer program kan en bruger tænde og slukke enhedder som er koblet til 230 VAC elnettet uden behov for ekstra kabler. En kodelås forhindrer børnende i selv at tænde for strømmen igen og forældre kan dermed slukke for farlige elektriske apparater centralt, så børn ikke kommer til skade hvis de ikke er under opsyn. 

Produktet er udviklet på de platforme der er gjort brug af på 1. og 2. semester. Som enhedder til modtagelse og afsendelse af data over elnettet bruges STK500 kittet fra Atmel samt egen udviklet hardware, til at interface til 230 VAC nettet og en almindelig PC bruges som bruger interface. Som kodelås anvendes DE2 boardet fra Altera.

