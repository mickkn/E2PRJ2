\chapter{Udviklingsværktøjer}
Gennem hele projektforløbet er der anvendt forskellige programmer og værktøjer til de respektive opgaver. Nogle programmer havde vi kendskab til på forhånd hvor andre var helt ny for enkelte eller alle gruppe medlemmer.

\section{LaTex}
Hele rapporten er skrevet i \LaTeX. Dette valg kom i starten af projektet da IDA havde et tilbud om et gratis endags kursus, hvor hele gruppen blev enige om at deltage. 

\LaTeX er et kodebaseret tekstredigerings program som er designet netop til større rapporter. Formålet er at gøre forfatteren fri for at skulle bekymre sig om formateringer således at han/hun kan rette al fokus på indholdet i rapporten.

Texmaker er benyttet som teksteditor.

Det krævede dog lidt tid i starten at komme i gang med \LaTeX, men da det var på plads fungerede det rigtig godt. 

\section{Visual Studio}

\section{Atmel Studio}

\section{Multisim}
Multisim er benyttet i forbindelse med degin af kredsløbsdiagrammer. 

\section{Microsoft Visio} % UML & SysML

\section{Quartus II}

\section{Filhåndtering}

\subsection{GitHub}


\subsection{Dropbox}

Dropbox benyttes som cloud løsning. Dropbox har fungeret som fælles harddisk. Primært benyttet i forbindelse med de 2 afholde reviews. Ydermere er dropbox benyttet til deling af litteratur. 

\subsection{Google Drev}

Logbogen, mødereferater er udarbejdet i Google Drevs dokument funktion. Og tidsplanen er udarbejdet i regneark funktionen. På den måde kan alle se og rette i det samme dokument samtidigt. 


