\chapter{Udviklingsværktøjer}
Gennem hele projektforløbet er der anvendt forskellige programmer og værktøjer til de respektive opgaver. Nogle programmer havde vi kendskab til på forhånd hvor andre var helt nye for enkelte eller alle gruppemedlemmer.

\section{LaTex og Texmaker (JS, PO)}
Hele rapporten er skrevet i \LaTeX. Dette valg kom i starten af projektet da IDA havde et tilbud om et gratis endags kursus, hvor hele gruppen blev enige om at deltage. 

\LaTeX er et kodebaseret tekstredigerings sprog som er designet netop til større rapporter. Formålet er at gøre forfatteren fri for at skulle bekymre sig om formateringer således at han/hun kan rette al fokus på indholdet i rapporten.

Texmaker er benyttet som teksteditor. Heri er alt \LaTeX koden skrevet. Programmet har indbygget pdf-viewer, der gør det muligt at se det endelige produkt vha. Texmakers kompiler. Texmaker er desuden udstyret med dansk stavekontrol.

Det krævede dog lidt tid i starten at komme i gang med \LaTeX, men da det var på plads fungerede det rigtig godt. 

\section{Microsoft Visual Studio (BS)}
Udviklingsprogrammet Microsoft Visual Studio 2012 er brugt til softwareprogrammeringen til PC.

\section{Atmel Studio (BS)}
Atmel Studio 6.1 er det brugte værktøj til programmering af software til CSS-hovedenheden og X10-udtaget.

\section{National Instruments Multisim (PO)}
National Instruments Multisim er benyttet i forbindelse med design af kredsløbsdiagrammer. 

\section{Microsoft Visio (JS)} % UML & SysML
Som del af ISE-undervisning er der blevet undervist og anvendt Microsoft Visio til udarbejdelse af diverse diagrammer. Herunder UML og SysML. SysML er anvendt til at designe blok- og internalblok-diagrammer for hardwaren. UML er anvendt til klasse- og sekvensdiagrammer. 

\section{Altera Quartus II (PO)}
Altera Quartus II er anvendt til VHDL programmeringen af DE2 kodelåsen.

\section{Electronic ToolBox App (JS)}
Electronic ToolBox er en applikation der er udviklet til IOS. Den indeholder informationer omkring det meste elektronik og værktøjer til diverse udregninger af kredsløbsdesign. Vi har primært anvendt det til udregninger på knækfrekvenser i forbindelse med høj-, lav- og båndpasfilter.  

\section{Filhåndtering (JS, PO)}
Til håndtering af filer er nedenstående 3 løsninger brugt. 

\subsection{GitHub}
GitHub er et sky-basseret versionsstyringsprogram. Det er brugt til de produktmæssige dokumentationer, dvs. software kode, hardware diagrammer og projektdokumentation samt projektrapporten.

\subsection{Dropbox}
Dropbox benyttes som cloud løsning. Dropbox har fungeret som fælles harddisk. Primært benyttet i forbindelse med de 2 afholde reviews. Ydermere er dropbox benyttet til deling af litteratur. 

\subsection{Google Drev}
Logbogen, mødereferater er udarbejdet i Google Drevs dokument funktion. Og tidsplanen er udarbejdet i regneark funktionen. På den måde kan alle se og rette i det samme dokument samtidigt. 


