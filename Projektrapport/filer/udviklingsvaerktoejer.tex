\chapter{Udviklingsværktøjer}
Gennem hele projektforløbet er der anvendt forskellige programmer og værktøjer til de respektive opgaver. Nogle programmer havde vi kendskab til på forhånd hvor andre var helt ny for enkelte eller alle gruppe medlemmer.

\section{LaTex}
Hele rapporten er skrevet i \LaTeX. Dette valg kom i starten af projektet da IDA havde et tilbud om et gratis endags kursus, hvor hele gruppen blev enige om at deltage. 

\LaTeX er et kodebaseret tekstredigerings program som er designet netop til større rapporter. Formålet er at gøre forfatteren fri for at skulle bekymre sig om formateringer således at han/hun kan rette al fokus på indholdet i rapporten.

Det krævede dog lidt tid i starten at komme i gang med \LaTeX, men da det var på plads fungerede det rigtig godt. 

\section{Visual Studio}

\section{Atmel Studio}

\section{Multisim}

\section{Microsoft Visio} % UML & SysML

\section{Quartus II}

\section{Filhåndtering}

